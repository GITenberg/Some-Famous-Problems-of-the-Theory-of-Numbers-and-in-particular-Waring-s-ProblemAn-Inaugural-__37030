% %%%%%%%%%%%%%%%%%%%%%%%%%%%%%%%%%%%%%%%%%%%%%%%%%%%%%%%%%%%%%%%%%%%%%%% %
%                                                                         %
% The Project Gutenberg EBook of Some Famous Problems of the Theory of    %
% Numbers and in particular Waring's Problem, by G. H. (Godfrey Harold) Hardy
%                                                                         %
% This eBook is for the use of anyone anywhere at no cost and with        %
% almost no restrictions whatsoever.  You may copy it, give it away or    %
% re-use it under the terms of the Project Gutenberg License included     %
% with this eBook or online at www.gutenberg.net                          %
%                                                                         %
%                                                                         %
% Title: Some Famous Problems of the Theory of Numbers and in particular Waring's Problem
%        An Inaugural Lecture delivered before the University of Oxford   %
%                                                                         %
% Author: G. H. (Godfrey Harold) Hardy                                    %
%                                                                         %
% Release Date: August 10, 2011 [EBook #37030]                            %
%                                                                         %
% Language: English                                                       %
%                                                                         %
% Character set encoding: ISO-8859-1                                      %
%                                                                         %
% *** START OF THIS PROJECT GUTENBERG EBOOK FAMOUS PROBLEMS OF THEORY OF NUMBERS ***
%                                                                         %
% %%%%%%%%%%%%%%%%%%%%%%%%%%%%%%%%%%%%%%%%%%%%%%%%%%%%%%%%%%%%%%%%%%%%%%% %

\def\ebook{37030}
%%%%%%%%%%%%%%%%%%%%%%%%%%%%%%%%%%%%%%%%%%%%%%%%%%%%%%%%%%%%%%%%%%%%%%
%%                                                                  %%
%% Packages and substitutions:                                      %%
%%                                                                  %%
%% book:     Required.                                              %%
%% inputenc: Standard DP encoding. Required.                        %%
%%                                                                  %%
%% ifthen:   Logical conditionals. Required.                        %%
%%                                                                  %%
%% amsmath:  AMS mathematics enhancements. Required.                %%
%% amssymb:  Additional mathematical symbols. Required.             %%
%%                                                                  %%
%% alltt:    Fixed-width font environment. Required.                %%
%% caption:  Caption enhancements. Required.                        %%
%% array:    Enhanced tabular features. Required.                   %%
%%                                                                  %%
%% geometry: Enhanced page layout package. Required.                %%
%% fancyhdr: Enhanced page headers. Required.                       %%
%% hyperref: Hypertext embellishments for pdf output. Required.     %%
%%                                                                  %%
%%                                                                  %%
%% Producer's Comments:                                             %%
%%                                                                  %%
%%   Changes are noted in this file in two ways.                    %%
%%   1. \DPtypo{}{} for typographical corrections, showing original %%
%%      and replacement text side-by-side.                          %%
%%   2. [** TN: Note]s for lengthier or stylistic comments.         %%
%%                                                                  %%
%% Compilation Flags:                                               %%
%%                                                                  %%
%%   The following behavior may be controlled by boolean flags.     %%
%%                                                                  %%
%%   ForPrinting (false by default):                                %%
%%   Compile a screen-optimized PDF file. Set to true for print-    %%
%%   optimized file (large text block, two-sided layout, black      %%
%%   hyperlinks).                                                   %%
%%                                                                  %%
%%   Both print and screen layout are relatively loose.             %%
%%                                                                  %%
%%                                                                  %%
%% PDF pages: 46 (if ForPrinting set to false)                      %%
%% PDF page size: 4.75 x 7" (non-standard)                          %%
%% PDF document info: filled in                                     %%
%%                                                                  %%
%%                                                                  %%
%% Command block:                                                   %%
%%                                                                  %%
%%     pdflatex x2                                                  %%
%%                                                                  %%
%%                                                                  %%
%%                                                                  %%
%% August 2011: pglatex.                                            %%
%%   Compile this project with:                                     %%
%%   pdflatex 37030-t.tex ..... TWO times                           %%
%%                                                                  %%
%%   pdfTeXk, Version 3.141592-1.40.3 (Web2C 7.5.6)                 %%
%%                                                                  %%
%%%%%%%%%%%%%%%%%%%%%%%%%%%%%%%%%%%%%%%%%%%%%%%%%%%%%%%%%%%%%%%%%%%%%%
\listfiles
\documentclass[12pt]{book}[2007/10/19]

\usepackage[latin1]{inputenc}[2008/03/30]

\usepackage{ifthen}[2001/05/26]  %% Logical conditionals

\usepackage{amsmath}[2000/07/18] %% Displayed equations
\usepackage{amssymb}[2009/06/22] %% and additional symbols

\usepackage{alltt}[1997/06/16]   %% boilerplate, credits, license

\usepackage[labelformat=empty]{caption}
\usepackage{array}[2008/09/09]   %% extended array/tabular features

% for running heads
\usepackage{fancyhdr}

%%%%%%%%%%%%%%%%%%%%%%%%%%%%%%%%%%%%%%%%%%%%%%%%%%%%%%%%%%%%%%%%%
%%%% Interlude:  Set up PRINTING (default) or SCREEN VIEWING %%%%
%%%%%%%%%%%%%%%%%%%%%%%%%%%%%%%%%%%%%%%%%%%%%%%%%%%%%%%%%%%%%%%%%

% ForPrinting=true (default)           false
% Asymmetric margins                   Symmetric margins
% Black hyperlinks                     Blue hyperlinks
% Start Preface, ToC, etc. recto       No blank verso pages
%
\newboolean{ForPrinting}

%% UNCOMMENT the next line for a PRINT-OPTIMIZED VERSION of the text %%
%\setboolean{ForPrinting}{true}

%% Initialize values to ForPrinting=false
\newcommand{\Margins}{hmarginratio=1:1}     % Symmetric margins
\newcommand{\HLinkColor}{blue}              % Hyperlink color
\newcommand{\PDFPageLayout}{SinglePage}
\newcommand{\TransNote}{Transcriber's Note}
\newcommand{\TransNoteCommon}{%
  Minor typographical corrections and presentational changes have
  been made without comment.
  \bigskip
}

\newcommand{\TransNoteText}{%
  \TransNoteCommon

  This PDF file is optimized for screen viewing, but may easily be
  recompiled for printing. Please see the preamble of the \LaTeX\
  source file for instructions.
}
%% Re-set if ForPrinting=true
\ifthenelse{\boolean{ForPrinting}}{%
  \renewcommand{\Margins}{hmarginratio=2:3} % Asymmetric margins
  \renewcommand{\HLinkColor}{black}         % Hyperlink color
  \renewcommand{\PDFPageLayout}{TwoPageRight}
  \renewcommand{\TransNote}{Transcriber's Note}
  \renewcommand{\TransNoteText}{%
    \TransNoteCommon

    This PDF file is optimized for printing, but may easily be
    recompiled for screen viewing. Please see the preamble of the
    \LaTeX\ source file for instructions.
  }
}{% If ForPrinting=false, don't skip to recto
  \renewcommand{\cleardoublepage}{\clearpage}
}
%%%%%%%%%%%%%%%%%%%%%%%%%%%%%%%%%%%%%%%%%%%%%%%%%%%%%%%%%%%%%%%%%
%%%%  End of PRINTING/SCREEN VIEWING code; back to packages  %%%%
%%%%%%%%%%%%%%%%%%%%%%%%%%%%%%%%%%%%%%%%%%%%%%%%%%%%%%%%%%%%%%%%%

\ifthenelse{\boolean{ForPrinting}}{%
  \setlength{\paperwidth}{8.5in}%
  \setlength{\paperheight}{11in}%
  \usepackage[body={5in,8in},\Margins]{geometry}[2010/09/12]
}{%
  \setlength{\paperwidth}{4.75in}%
  \setlength{\paperheight}{7in}%
  \raggedbottom
  \usepackage[body={4.5in,6in},\Margins,includeheadfoot]{geometry}[2010/09/12]
}

\providecommand{\ebook}{00000}    % Overridden during white-washing
\usepackage[pdftex,
  hyperfootnotes=false,
  pdftitle={The Project Gutenberg eBook \#\ebook: Some Famous Problems of the Theory of Numbers and in particular Waring's Problem},
  pdfauthor={G. H. Hardy},
  pdfkeywords={Brenda Lewis, Anna Hall,
               Project Gutenberg Online Distributed Proofreading Team},
  pdfstartview=Fit,    % default value
  pdfstartpage=1,      % default value
  pdfpagemode=UseNone, % default value
  bookmarks=true,      % default value
  linktocpage=false,   % default value
  pdfpagelayout=\PDFPageLayout,
  pdfdisplaydoctitle,
  pdfpagelabels=true,
  bookmarksopen=true,
  bookmarksopenlevel=1,
  colorlinks=true,
  linkcolor=\HLinkColor]{hyperref}[2011/04/17]

%%%% Fixed-width environment to format PG boilerplate %%%%
\newenvironment{PGtext}{%
\begin{alltt}
\fontsize{8.1}{9}\ttfamily\selectfont}%
{\end{alltt}}

% Running heads
\newcommand{\FlushRunningHeads}{\clearpage\fancyhf{}\cleardoublepage}

\newcommand{\BookMark}[2]{\phantomsection\pdfbookmark[#1]{#2}{#2}}

%% Major document divisions %%
\newcommand{\PGBoilerPlate}{%
  \pagenumbering{Alph}
  \pagestyle{empty}
  \BookMark{0}{PG Boilerplate.}
}
\newcommand{\FrontMatter}{%
  \cleardoublepage
  \frontmatter
  \pagestyle{empty}
}

\newcommand{\MainMatter}{%
  \FlushRunningHeads
  \mainmatter
  \pagestyle{fancy}
  \fancyhf{}
  \setlength{\headheight}{15pt}
  \thispagestyle{plain}
  \fancyhead[CE]{SOME FAMOUS PROBLEMS OF}
  \fancyhead[CO]{THE THEORY OF NUMBERS}
  \renewcommand{\headrulewidth}{0pt}

  \ifthenelse{\boolean{ForPrinting}}
             {\fancyhead[RO,LE]{\thepage}}
             {\fancyhead[R]{\thepage}}
}

\newcommand{\BackMatter}{%
  \FlushRunningHeads
  \backmatter
  \BookMark{-1}{Back Matter.}
}

%%Tables
\newcommand{\Table}[1]{%
  \phantomsection\label{table:#1}%
  \caption{TABLE #1}%
}

\newcommand{\TableRef}[2]{%
\hyperref[table:#2]{#1}%
}

\DeclareInputMath{183}{\cdot}

\newcommand{\Title}[1]{\section*{\centering #1}}
\newcommand{\First}[1]{\textsc{#1}}

\newlength{\TmpLen}
% For Table I. \Tally[sign]{number}; sign defaults to +
\newcommand{\Tally}[2][+]{%
  \settowidth{\TmpLen}{\ensuremath{99,999.999}}%
  \makebox[\TmpLen][s]{$#1$\hfill$#2$}%
}

\newcommand{\ColHead}[1]{\multicolumn{1}{c}{#1}}
\newcommand{\EmDash}{\text{---}}
\newcommand{\EnDash}{\text{--}}

\newcommand{\Thint}{\makebox[8pt][c]{\ensuremath{\displaystyle\int}}}
\newcommand{\iiiiint}{\;\Thint\Thint\Thint\Thint\Thint\;}

\newcommand{\DPtypo}[2]{#2}
\newcommand{\DPnote}[1]{}

\renewcommand{\leq}{\leqq}
\renewcommand{\geq}{\geqq}
\renewcommand{\epsilon}{\varepsilon}
\renewcommand{\S}{\raisebox{-0.5ex}{\textbf{\large S}}}

%%%%%%%%%%%%%%%%%%%%%%%% START OF DOCUMENT %%%%%%%%%%%%%%%%%%%%%%%%%%

\begin{document}
\PGBoilerPlate
\small
\begin{PGtext}
The Project Gutenberg EBook of Some Famous Problems of the Theory of
Numbers and in particular Waring's Problem, by G. H. (Godfrey Harold) Hardy

This eBook is for the use of anyone anywhere at no cost and with
almost no restrictions whatsoever.  You may copy it, give it away or
re-use it under the terms of the Project Gutenberg License included
with this eBook or online at www.gutenberg.net


Title: Some Famous Problems of the Theory of Numbers and in particular Waring's Problem
       An Inaugural Lecture delivered before the University of Oxford

Author: G. H. (Godfrey Harold) Hardy

Release Date: August 10, 2011 [EBook #37030]

Language: English

Character set encoding: ISO-8859-1

*** START OF THIS PROJECT GUTENBERG EBOOK FAMOUS PROBLEMS OF THEORY OF NUMBERS ***
\end{PGtext}
\clearpage
\begin{PGtext}
Produced by Anna Hall, Brenda Lewis and the Online
Distributed Proofreading Team at http://www.pgdp.net (This
file was produced from images generously made available
by The Internet Archive/American Libraries.)
\end{PGtext}
\vfill

\BookMark{0}{Transcriber's Note}
\subsection*{\centering\normalfont\scshape
\normalsize\MakeLowercase{\TransNote}}

\TransNoteText
\clearpage

%%%%%%%%%%%%%%%%%%%%%%%%%%% FRONT MATTER %%%%%%%%%%%%%%%%%%%%%%%%%%

\FrontMatter
%% -----File: 001.png---
\begin{center}

\textbf{\LARGE SOME FAMOUS PROBLEMS} \\
\vfill

of the \\[12pt]
\vfill

\textbf{\LARGE THEORY OF NUMBERS} \\
\vfill

and in particular \\[12pt]
\vfill

{\LARGE Waring's Problem} \\
\vfill
\vfill

\textit{An Inaugural Lecture delivered before the \\
University of Oxford}
\vfill

{\scriptsize BY} \\
\vfill

\textbf{G.~H. HARDY, M.A., F.R.S\@.} \\
\medskip
{\itshape\scriptsize Fellow of New College \\
Savilian Professor of Geometry in the University of Oxford \\
and late Fellow of Trinity College, Cambridge\par}
\vfill
\vfill

OXFORD \\
AT THE CLARENDON PRESS \\
1920
\end{center}
%% -----File: 002.png---
%[Blank Page]
%% -----File: 003.png---
\iffalse %%duplicate of previous page
SOME  FAMOUS  PROBLEMS
of the
THEORY OF NUMBERS
and in particular
Waring's  Problem

\textit{An Inaugural Lecture delivered before the
University of Oxford}

BY

G.~H. HARDY, M.A., F.R.S\@.

\textit{Fellow of New College
Savilian Professor of Geometry in the University of Oxford
and late Fellow of Trinity College, Cambridge}

OXFORD

AT THE  CLARENDON  PRESS

1920
\fi
%% -----File: 004.png---
\newpage
\begin{center}
\null\vfill

OXFORD UNIVERSITY PRESS \\
{\scriptsize LONDON EDINBURGH GLASGOW NEW YORK \\
TORONTO MELBOURNE CAPE TOWN BOMBAY\par}
\smallskip
HUMPHREY MILFORD \\
{\scriptsize PUBLISHER TO THE UNIVERSITY}
\vfill

\end{center}
%% -----File: 005.png---
\MainMatter
\Title{SOME FAMOUS PROBLEMS OF THE
THEORY OF NUMBERS.}

\First{It} is expected that a professor who delivers an inaugural
lecture should choose a subject of wider interest than those
which he expounds to his ordinary classes. This custom is
entirely reasonable; but it leaves a pure mathematician
faced by a very awkward dilemma. There are subjects in
which only what is trivial is easily and generally comprehensible.
Pure mathematics, I am afraid, is one of them;
indeed it is more: it is perhaps the one subject in the
world of which it is true, not only that it is genuinely
difficult to understand, not only that no one is ashamed of
inability to understand it, but even that most men are
more ready to exaggerate than to dissemble their lack of
understanding.

There is one method of meeting such a situation which
is sometimes adopted with considerable success. The
lecturer may set out to justify his existence by enlarging
upon the overwhelming importance, both to his University
and to the community in general, of the particular studies on
which he is engaged. He may point out how ridiculously
inadequate is the recognition at present afforded to them;
how urgent it is in the national interest that they should be
largely and immediately re-endowed; and how immensely
all of us would benefit were we to entrust him and his
colleagues with a predominant voice in all questions of
educational administration. I have observed friends of my
own, promoted to chairs of various subjects in various
%% -----File: 006.png---
Universities, addressing themselves to this task with an
eloquence and courage which it would be impertinent in me
to praise. For my own part, I trust that I am not lacking
in respect either for my subject or myself. But, if I am
asked to explain how, and why, the solution of the problems
which occupy the best energies of my life is of
importance in the general life of the community, I must
decline the unequal contest: I have not the effrontery to
develop a thesis so palpably untrue. I must leave it to the
engineers and the chemists to expound, with justly prophetic
fervour, the benefits conferred on civilization by
gas-engines, oil, and explosives. If I could attain every
scientific ambition of my life, the frontiers of the Empire
would not be advanced, not even a black man would be
blown to pieces, no one's fortune would be made, and least
of all my own. A pure mathematician must leave to
happier colleagues the great task of alleviating the sufferings
of humanity.

I suppose that every mathematician is sometimes depressed,
as certainly I often am myself, by this feeling of
helplessness and futility. I do not profess to have any
very satisfactory consolation to offer. It is possible that
the life of a mathematician is one which no perfectly
reasonable man would elect to live. There are, however,
one or two reflections from which I have sometimes found
it possible to extract a certain amount of comfort. In the
first place, the study of mathematics is, if an unprofitable,
a perfectly harmless and innocent occupation, and we have
learnt that it is something to be able to say that at any
rate we do no harm. Secondly, the scale of the universe
is large, and, if we are wasting our time, the waste of the
lives of a few university dons is no such overwhelming
catastrophe. Thirdly, what we do may be small, but it
has a certain character of permanence; and to have produced
%% -----File: 007.png---
anything of the slightest permanent interest, whether
it be a copy of verses or a geometrical theorem, is to have
done something utterly beyond the powers of the vast
majority of men. And, finally, the history of our subject
does seem to show conclusively that it is no such mean
study after all. The mathematicians of the past have not
been neglected or despised; they have been rewarded in a
manner, undiscriminating perhaps, but certainly not ungenerous.
At all events we can claim that, if we are
foolish in the object of our devotion, we are only in our
small way aping the folly of a long line of famous men,
and that, in these days of conflict between ancient and
modern studies, there must surely be something to be said
for a study which did not begin with Pythagoras, and will
not end with Einstein, but is the oldest and the youngest
of all.

It seemed to me for a moment, when I was considering
what subject I should choose, that there was perhaps one
which might, in a philosophic University like this, be of
wider interest than ordinary technical mathematics. If
modern pure mathematics has any important applications,
they are the applications to philosophy made by the mathematical
logicians of the last thirty years. In the sphere of
philosophy we mathematicians put forward a strictly
limited but absolutely definite claim. We do not claim
that we hold in our hands the key to all the riddles of
existence, or that our mathematics gives us a vision of
reality to which the less fortunate philosopher cannot
attain; but we do claim that there are a number of
puzzles, of an abstract and elusive kind, with which the
philosophers of the past have struggled ineffectually, and
of which we now can give a quite definite and explicit
solution. There is a certain region of philosophical territory
which it is our intention to annex. It is a strictly
%% -----File: 008.png---
demarcated region, but it has suffered under the misrule of
philosophers for generations, and it is ours by right; we
propose to accept the mandate for it, and to offer it the
opportunity of self-determination under the mathematical
flag. Such at any rate is the thesis which I hope it may
before long be my privilege to defend.

It seemed to me, however, when I considered the matter
further, that there are two fatal objections to mathematical
philosophy as a subject for an inaugural address. In the
first place the subject is one which requires a certain
amount of application and preliminary study. It is not
that it is a subject, now that the foundations have been
laid, of any extraordinary difficulty or obscurity; nor that
it demands any wide knowledge of ordinary mathematics.
But there are certain things that it does demand; a little
thought and patience, a little respect for mathematics, and
a little of the mathematical habit of mind which comes
fully only after long years spent in the company of mathematical
ideas. Something, in short, may be learnt in a
term, but hardly in a casual hour.

In the second place, I think that a professor should
choose, for his inaugural lecture, a subject, if such a subject
exists, to which he has made himself some contribution of
substance and about which he has something new to say.
And about mathematical philosophy I have nothing new to
say; I can only repeat what has been said by the men,
Cantor and Frege in Germany, Peano in Italy, Russell and
Whitehead in England, who have originated the subject
and moulded it now into something like a definite form.
It would be an insult to my new University to offer it
a watered synopsis of some one else's work. I have therefore
finally decided, after much hesitation, to take a subject
which is quite frankly mathematical, and to give a
summary account of the results of some researches which,
%% -----File: 009.png---
whether or no they contain anything of any interest or
importance, have at any rate the merit that they represent
the best that I can do.

My own favourite subject has certain redeeming advantages.
It is a subject, in the first place, in which a large
proportion of the most remarkable results are by no means
beyond popular comprehension. There is nothing in the
least popular about its \emph{methods}; as to its votaries it is the
most beautiful, so by common consent it is the most difficult
of all branches of a difficult science; but many of the
actual results are such as can be stated in a simple and
striking form. The subject has also a considerable historical
connexion with this particular chair. I do not wish to
exaggerate this connexion. It must be admitted that the
contributions of English mathematicians to the Theory of
Numbers have been, in the aggregate, comparatively slight.
Fermat was not an Englishman, nor Euler, nor Gauss, nor
Dirichlet, nor Riemann; and it is not Oxford or Cambridge,
but G�ttingen, that is the centre of arithmetical research
to-day. Still, there has been an English connexion, and it
has been for the most part a connexion with Oxford and
with the Savilian chair.

The connexion of Oxford with the theory of numbers is
in the main a nineteenth-century connexion, and centres
naturally in the names of Sylvester and Henry Smith.
There is a more ancient, if indirect, connexion which I
ought not altogether to forget. The theory of numbers,
more than any other branch of pure mathematics, has
begun by being an empirical science. Its most famous
theorems have all been conjectured, sometimes a hundred
years or more before they have been proved; and they
have been suggested by the evidence of a mass of computation.
Even now there is a considerable part to be
played by the computer; and a man who has to undertake
%% -----File: 010.png---
laborious arithmetical computations is hardly likely to
forget what he owes to Briggs. However, this is ancient
history, and it is with Sylvester and Smith that I am
concerned to-day, and more particularly with Smith.

Henry Smith was very many things, but above all things
a most brilliant arithmetician. Three-quarters of the first
volume of his memoirs is occupied with the theory of
numbers, and Dr.~Glaisher, his mathematical biographer,
has observed very justly that, even when he is primarily
concerned with other matters, the most striking feature of
his work is the strongly arithmetical spirit which pervades
the whole. His most remarkable contributions to the
theory are contained in his memoirs on the arithmetical
theory of forms, and in particular in the famous memoir on
the representation of numbers by sums of five squares,
crowned by the Paris Academy and published only after
his death. This memoir is peculiarly interesting to me, for
the problem is precisely one of those of which I propose to
speak to-day; and I may perhaps add one comment on the
surprising history set out in Dr.~Glaisher's introduction.
The name of Minkowski is familiar to-day to many, even
in Oxford, who have certainly never read a line of Smith.
It is curious to contemplate at a distance the storm of
indignation which convulsed the mathematical circles of
England when Smith, bracketed after his death with the
then unknown German mathematician, received a greater
honour than any that had been paid to him in life.

The particular problems with which I am concerned
belong to what is called the `additive' side of higher
arithmetic. The general problem may be stated as follows.

Suppose that $n$~is any positive integer, and
\[
\alpha_{1},\thickspace \alpha_{2},\thickspace \alpha_{3},\thickspace \dots
\]
positive integers of some special kind, squares, for example,
%% -----File: 011.png---
or cubes, or perfect $k$th~powers, or primes. We consider all
possible expressions of~$n$ in the form
\[
n = \alpha_{1} + \alpha_{2} + \dots + \alpha_{s},
\]
where $s$~may be fixed or unrestricted, the $\alpha$'s may or may
not be necessarily distinct, and order may or may not be
relevant, according to the particular problem on which we
are engaged. We denote by
\[
r(n)
\]
the number of representations which satisfy the conditions
of the problem. Then \emph{what can we say about~$r(n)$}? Can
we find an exact formula for~$r(n)$, or an approximate formula
valid for large values of~$n$? In particular, is $r(n)$ \emph{always
positive}? Is it always possible, that is to say, to find at
least \emph{one} representation of~$n$ of the type required? Or, if
this is not so, is it at any rate always possible when $n$~is
sufficiently large?

I can illustrate the nature of the general problem most
simply by considering for a moment an entirely trivial
case. Let us suppose that there are three different~$\alpha$'s only,
viz.\ the numbers $1$,~$2$,~$3$; that repetitions of the same~$\alpha$ are
permissible; that the order of the~$\alpha$'s is irrelevant; and
that~$s$, the number of the~$\alpha$'s, is unrestricted. Then it is
easy to see that $r(n)$, the number of representations, is the
number of solutions of the equation
\[
n = x + 2y + 3z
\]
in positive integers, including zero.

There are various ways of solving this extremely simple
problem. The most interesting for our present purpose is
that which rests on an analytical foundation, and uses the
idea of the \emph{generating function}
\[
f(x) = 1 + \sum_{1}^{\infty} r(n)x^{n},
\]
%% -----File: 012.png---
in which the coefficients are the values of the arithmetical
function~$r(n)$. It follows immediately from the definition
of~$r(n)$ that
\begin{multline*}
f(x) = (1 + x + x^{2} + \dots)(1 + x^{2} + x^{4} + \dots)(1 + x^{3} + x^{6} + \dots)\\
     = \frac{1}{(1 - x)(1 - x^{2})(1 - x^{3})};
\end{multline*}
and, in order to determine the coefficients in the expansion,
nothing more than a little elementary algebra is required.
We find, by the ordinary theory of partial fractions, that
\begin{multline*}
f(x) = \frac{1}{6(1 - x)^{3}} + \frac{1}{4(1 - x)^{2}}
     + \frac{17}{72(1 - x)} + \frac{1}{8(1 + x)}\\
     + \frac{1}{9(1 - \omega x)} + \frac{1}{9(1 - \omega^{2} x)},
\end{multline*}
where $\omega$~and~$\omega^{2}$ denote as usual the two complex cube roots
of unity. Expanding the fractions, and picking out the
coefficient of~$x^{n}$, we obtain
\[
r(n) = \frac{(n + 3)^{2}}{12} - \frac{7}{72} + \frac{(-1)^{n}}{8} + \frac{2}{9} \cos \frac{2n\pi}{3}.
\]
It is easily verified that the sum of the last three terms
can never be as great as~$\frac{1}{2}$, so that $r(n)$~is the integer
nearest to
\[
\frac{(n + 3)^{2}}{12}.
\]

The problem is, as I said, quite trivial, but it is interesting
none the less. A great deal of work has been done
on problems similar in kind, though naturally far more
complex and difficult in detail, by Cayley and Sylvester,
for example, in the last century, and by Glaisher, and
above all by \DPtypo{Macmahon}{MacMahon}, in this. And even this problem,
%% -----File: 013.png---
simple as it is, has sufficient content to bring out clearly
certain principles of cardinal importance.

In particular, the solution of the problem shows quite
clearly that, if we are to attack these `additive' problems
by analytic methods, it is in the theory of integral power
series
\[
  \sum a_{n} x^{n}
\]
that the necessary machinery must be found. It is this
characteristic which distinguishes this theory sharply from
the other great side of the analytic theory of numbers, the
`multiplicative' theory, in which the fundamental idea is
that of the resolution of a number into primes. In the
latter theory the right weapon is generally not a power
series, but what is called a Dirichlet's series, a series of the
type
\[
  \sum a_{n} n^{-s}.
\]
It is easy to see this by considering a simple example.
One of the most interesting functions of the multiplicative
theory is~$d(n)$, the number of divisors of~$n$. The associated
power series
\[
 \sum d(n) x^{n}
\]
is easily transformed into the series
\[
 \sum \frac{x^{n}}{1 - x^{n}},
\]
called Lambert's series. The function is an interesting
one, but somewhat unmanageable, and certainly not one of
the fundamental functions of analysis. The corresponding
Dirichlet's series is far more fundamental; it is in fact
\[
  \sum \frac{d(n)}{n^{s}} = \left(\sum \frac{1}{n^{s}}\right)^{2}
    = \Big(\zeta (s)\Big)^{2},
\]
the square of the famous Zeta function of Riemann.
%% -----File: 014.png---

The underlying reason for this distinction is fairly
obvious. It is natural to \emph{multiply} primes and unnatural
to \emph{add} them. Now
\[
  m^{-s} � n^{-s} = (mn)^{-s},
\]
so that, in the theory of Dirichlet's series, the terms
combine naturally with one another in a `multiplicative'
manner. But
\[
  x^{m} � x^{n} = x^{m + n},
\]
so that the multiplication of two terms of a power series
involves an additive operation on their ranks. It is
thus that the Dirichlet's series rather than the power
series proves to be the proper weapon in the theory of
primes.

It would be difficult for anybody to be more profoundly
interested in anything than I am in the theory of primes;
but it is not of this theory that I propose to speak to-day,
and we must return to our general additive problem. As
soon as we try to specialize the problem in some more
interesting manner, two problems stand out as calling for
research. Each of them, naturally, is only the representative
of a class.

The first of these problems is the \emph{problem of partitions}.
Let us suppose now that the $\alpha$'s are \emph{any} positive integers,
and that (as in the trivial problem) repetitions are allowed,
order is irrelevant, and $s$~is unrestricted. The problem is
then that of expressing~$n$ in any manner as a sum of
integral parts, or of solving the equation
\[
  n = x + 2y + 3z + 4u + 5v + \dots,
\]
and $r(n)$ or, as it is now more naturally written, $p(n)$, is
the number of \emph{unrestricted partitions} of~$n$. Thus
\begin{multline*}
  5 = 1 + 1 + 1 + 1 + 1 = 1 + 1 + 1 + 2\\
    = 1 + 2 + 2 = 1 + 1 + 3 = 2 + 3 = 1 + 4 = 5,
\end{multline*}
%% -----File: 015.png---
so that $p(5) = 7$. The generating function in this case
was found by Euler, and is
\[
  f(x) = 1 + \sum_{1}^{\infty} p(n) x^{n}
    = \frac{1}{(1 - x) (1 - x^{2}) (1 - x^{3}) \dots}.
\]

I do not wish to discuss this problem in any detail
now, but the form of the generating function calls for
one or two general remarks. In the trivial problem the
generating function was \emph{rational}, with a finite number of
poles all situated upon the unit circle. Here also we are
led to a power series, or infinite product, convergent inside
the unit circle; but there the resemblance ends. This
function will be recognized by any one familiar with the
theory of elliptic functions; it is an elliptic modular
function; and, like all such functions, it has the unit circle
as a continuous line of singularities and does not exist at
all outside. It is easy to imagine the immensely increased
difficulties of any analytic solution of the problem.

It was conjectured by a very brilliant Hungarian
mathematician, Mr.~G. P�lya, five or six years ago, that
\emph{any} function represented by a power series whose
coefficients are \emph{integers}, and which is convergent inside
the unit circle, must behave, in this respect, like one or
other of the two generating functions which we have
considered. Either such a function is a rational function,
that is to say, completely elementary; or else the unit
circle is a line of essential singularities. I believe that a
proof of this theorem has now been found by Mr.~F. Carlson
of Upsala, and is to be published shortly in the \textit{Mathematische
Zeitschrift}. It is difficult for me to give reasoned
praise to a memoir which I have not seen, but I can
recommend the theorem to your attention with confidence
as one of the most beautiful of recent years.

The problem of partitions is one to which, in collaboration
with the Indian mathematician, Mr.~S. Ramanujan, I have
%% -----File: 016.png---
myself devoted a great deal of work. The principal result
of our work has been the discovery of an approximate
formula for~$p(n)$ in which the leading term is
\[
\frac{1}{2\pi\sqrt{2}}
\frac{d}{dn}
\frac{e^{\frac{2\pi}{\sqrt{6}}\sqrt{n - \smash[b]{\frac{1}{24}}}}}
     {\sqrt{n - \smash[b]{\frac{1}{24}}}},
\]
and which enables us to approximate to~$p(n)$ with an
accuracy which is almost uncanny. We are able, for
example, by using $8$~terms of our formula, to calculate
$p(200)$, a number of $13$~figures, with an error of~$.004$.
I have set out the details of the calculation in \TableRef{Table~I}{I.}\@.
\begin{table}[hbt!]
\Table{I.}
\[
\begin{array}{r}
\ColHead{p(200)}\\
3,972,998,993,185.896\\
36,282.978\\
\Tally[-]{87.555}\\
\Tally{5.147}\\
\Tally{1.424}\\
\Tally{0.071}\\
\Tally{0.000}\\
\Tally{0.043}\\
\hline
3,972,999,029,388.004\\
\hline
\end{array}
\]
\end{table}
The value of~$p(200)$ was subsequently verified by Major
MacMahon, by a direct computation which occupied over
a month.

The formulae connected with this problem are very
elaborate, and except on the purely numerical side, where
the results of the theory are compared with those of computation,
it is not very well suited for a hasty exposition;
and I therefore pass on at once to the principal
object of my lecture, the very famous problem known,
after a Cambridge professor of the eighteenth century, as
\emph{Waring's Problem}.
%% -----File: 017.png---

We suppose now that every~$\alpha$ is a perfect $k$-th~power
$m^{k}$, $k$~being fixed in each case of the problem which we
consider; $m$~may be of either sign if $k$~is even, but must
be positive if $k$~is odd. In either case we allow $m$ to be
zero. Repetitions are permitted, as in our previous problems;
but it is more convenient now to take account of the order
of the~$\alpha$'s; and $s$, which was formerly unrestricted, is now
fixed in each case of the problem, like~$k$. The problem is
therefore that of determining the number of representations
of a number~$n$ as the sum of $s$~positive $k$-th~powers. Thus
Henry Smith's problem, the problem of five squares, is the
particular case of Waring's problem in which $k$~is~$2$ and
$s$~is~$5$. The problem has a long history, which centres
round this simplest case of squares; a history which began,
I suppose, with the right-angled triangles of Pythagoras,
and has been continued by a long succession of mathematicians,
including Fermat, Euler, Lagrange, and Jacobi, down
to the present day. I will begin by a summary of what
is known in the simplest case, where the solution is
practically complete.

A number~$n$ is the sum of \textbf{two} squares if and only if
it is of the form
\[
n = M^{2}P,
\]
where $P$~is a product of primes, all different and all of the
form~$4k + 1$. In particular, a prime number of the form
$4k + 1$ can be expressed as the sum of two squares, and
substantially in only one way. Thus $5 = 1^{2} + 2^{2}$, and there
is no other solution except the solutions $(�1)^{2} + (�2)^{2}$,
$(�2)^{2} + (�1)^{2}$, which are not essentially different, although
it is convenient to count them as distinct. The number of
numbers less than~$x$, and expressible as the sum of two
squares, is approximately
\[
  \frac{Cx}{\sqrt{\log x}},
\]
%% -----File: 018.png---
where $C$~is a certain constant. The last result was proved
by Landau in~1908; all the rest belong to the classical
theory.

A number is the sum of \textbf{three} squares unless it is of
the form
\[
4^{\alpha}(8k + 7),
\]
when it is not so expressible. \emph{Every} number may be
expressed by \textbf{four} squares, and \textit{a~fortiori} by five or more.
It is this last theorem of Lagrange that I would ask you
particularly to bear in mind.

If $s$, the number of squares, is even and less than~$10$,
the number of representations may be expressed in a very
simple form by means of the divisors of~$n$. Thus the
number of representations by $4$~squares, when $n$~is odd, is
$8$~times the sum of the divisors of~$n$; when $n$~is even, it
is $24$~times the sum of the odd divisors; and there are
similar results for $2$~squares, or~$6$, or~$8$.

When $s$~is $3$,~$5$, or~$7$, the number of representations can
also be found in a simple form, though one of a very
different character. Suppose, for example, that $s$~is~$3$. The
problem is in this case essentially the same as that of
determining the number of classes of binary arithmetical
forms of determinant~$-n$; and the solution depends on
certain finite sums of the form
\[
\sum \beta,\quad \sum \gamma,
\]
extended over quadratic residues~$\beta$ or non-residues~$\gamma$ of~$n$.

When $s$, whether even or odd, is greater than~$8$, the
solution is decidedly more difficult, and it is only very
recently that a uniform method of solution, for which I
must refer you to some recent memoirs of Mr.~L.~J. Mordell
and myself, has been discovered. For the moment I wish
to concentrate your attention on two points: the first, that
%% -----File: 019.png--- %%moved up one line to avoid breaking emph.
\emph{an expression by $4$~squares is always possible, while one
by~$3$ is not}; and the second, that the existence of numbers
not expressible by $3$~squares is revealed at once by the quite
trivial observation that no number so expressible can be
congruent to~$7$ to modulus~$8$.

It is plain, when we proceed to the general case, that
any number~$n$ can be expressed as a sum of $k$-th~powers;
we have only to take, for example, the sum of $n$~ones. And,
when $n$~is given, there is a \emph{minimum} number of $k$-th~powers
in terms of which $n$~can be expressed; thus
\[
23 = 2�2^{3} + 7�1^{3}
\]
is the sum of $9$~cubes and of no smaller number. But it is
not at all plain (and this is the point) that this minimum
number cannot tend to infinity with~$n$. It does not when
$k = 2$; for then it cannot exceed~$4$. And Waring's Problem
(in the restricted sense in which the name has commonly
been used) is the problem of proving that the minimum
number is similarly bounded in the general case. It is not
an easy problem; its difficulty may be judged from the fact
that it took 127~years to solve.

We may state the problem more formally as follows.
Let $k$ be given. Then there may or may not exist a
number~$m$, the same for all values of~$n$, and such that
$n$~can always be expressed as the sum of~$m$ $k$-th~powers or
less. If any number~$m$ possesses this property, all larger
numbers plainly possess it too; and among these numbers
we may select the \emph{least}. This least number, which will
plainly depend on~$k$, we call~$g(k)$; thus $g(k)$ is, by definition,
the least number, if such a number exists, for which
it is true that
\begin{quotation}
\textit{`every number is the sum of~$g(k)$ $k$-th~powers or less'.}
\end{quotation}
We have seen already that $g(2)$~exists and has the value~$4$.

In the third edition of his \textit{Meditationes Algebraicae},
published in Cambridge in~1782, Waring asserted that
%% -----File: 020.png---
every number is the sum of not more than $4$~squares, not
more than $9$~cubes, not more than $19$~fourth powers, \textit{et sic
deinceps}. A little more precision would perhaps have
been desirable; but it has generally been held, and I do
not question that it is true, that what Waring is asserting
is precisely the existence of~$g(k)$. He implies, moreover,
that $g(2) = 4$ and $g(3) = 9$; and both of these assertions
are correct, though in the first he had been anticipated by
Lagrange. Whether $g(4)$ is or is not equal to~$19$ is not
known to-day.

Waring advanced no argument of any kind in support
of his assertion, and it is in the highest degree unlikely
that he was in possession of any sort of proof. I have no
desire to detract from the reputation of a man who was
a very good mathematician if not a great one, and who
held a very honourable position in a University which not
even Oxford has persuaded me entirely to forget. But
there is a tendency to exaggerate the profundity implied
by the enunciation of a theorem of this particular kind.
We have seen this even in the case of Fermat, a mathematician
of a class to which Waring had not the slightest
pretensions to belong, whose notorious assertion concerning
`Mersenne's numbers' has been exploded, after the lapse of
over 250~years, by the calculations of the American computer
Mr.~Powers. No very laborious computations would be
necessary to lead Waring to a highly plausible speculation,
which is all I take his contribution to the theory to be;
and in the Theory of Numbers it is singularly easy to
speculate, though often terribly difficult to prove; and it
is only proof that counts.

The next advance towards the solution of the problem
was made by Liouville, who established the existence of~$g(4)$.
Liouville's proof, which was first published in~1859,
is quite simple and, as the simplest example of an important
%% -----File: 021.png---
type of argument, is worth reproducing here. It may be
verified immediately that
\begin{multline*}
6X^{2} = 6(x^{2} + y^{2} + z^{2} + t^{2})^{2}\\
\begin{aligned}
&= (x + y)^{4} + (x - y)^{4} + (z + t)^{4} + (z - t)^{4}\\
&+ (x + z)^{4} + (x - z)^{4} + (t + y)^{4} + (t - y)^{4}\\
&+ (x + t)^{4} + (x - t)^{4} + (y + z)^{4} + (y - z)^{4};
\end{aligned}
\end{multline*}
and since, by Lagrange's theorem, any number~$X$ is the
sum of $4$~squares, it follows that any number of the form~$6X^{2}$
is the sum of $12$~biquadrates. Hence any number of
the form $6(X^{2} + Y^{2} + Z^{2} + T^{2})$ or, what is the same thing,
any number of the form~$6m$, is the sum of $48$~biquadrates.
But \emph{any} number~$n$ is of the form~$6m + r$, where $r$~is
$0$,~$1$, $2$, $3$, $4$, or~$5$. And therefore $n$~is, at worst, the sum
of $53$~biquadrates. That is to say, $g(4)$~exists, and does
not exceed~$53$. Subsequent investigators, refining upon this
argument, have been able to reduce this number to~$37$; the
final proof that $g(4) \leq 37$, the most that is known at present,
was given by Wieferich in~1909. The number
\[
79 = 4�2^{4} + 15�1^{4}
\]
needs $19$~biquadrates, and no number is known which needs
more. There is therefore still a wide margin of uncertainty
as to the actual value of~$g(4)$.

The case of cubes is a little more difficult, and the
existence of~$g(3)$ was not established until 1895, when
Maillet proved that $g(3) \leq 17$. The proof then given by
Maillet rests upon the identity
\begin{multline*}
6x(x^{2} + y^{2} + z^{2} + t^{2})\\
= (x + y)^{3} + (x - y)^{3} + (x + z)^{3} + (x - z)^{3} + (x + t)^{3} + (x - t)^{3},
\end{multline*}
and the known results concerning the expression of a
number by $3$~squares. It has not the striking simplicity
of Liouville's proof; but it has enabled successive investigators
to reduce the number of cubes, until finally Wieferich,
%% -----File: 022.png---
in~1909, proved that $g(3) \leq 9$. As $23$~and~$239$ require
$9$~cubes, the value of~$g(3)$ is in fact exactly~$9$. It is only
for $k = 2$ and $k = 3$ that the actual value of~$g(k)$ has
been determined. But similar existence proofs were found,
and upper bounds for~$g(k)$ determined, by various writers,
in the cases $k = 5$,~$6$, $7$, $8$, and~$10$.

Before leaving the problem of the cubes I must call your
attention to another very beautiful theorem of a slightly
different character. The numbers $23$~and~$239$ require
$9$~cubes, and it appears, from the results of a survey of
all numbers up to~$40,000$, that no other number requires so
many. It is true that this has not actually been proved;
but it \emph{has} been proved (and this is of course the essential
point) that the number of numbers which require as many
cubes as~$9$ is \emph{finite}.

This singularly beautiful theorem, which is due to my
friend Professor Landau of G�ttingen, and is to me as
fascinating as anything in the theory, also dates from
1909, a year which stands out for many reasons in the
history of the problem. It is of exceptional interest not
only in itself but also on account of the method by which
it was proved, which utilizes some of the deepest results in
the modern theory of the asymptotic distribution of primes,
and made it, until very recently, the only theorem of its
kind erected upon a genuinely transcendental foundation.
To me it has a personal interest also, as being the only
theorem of the kind which, up to the present, defeats
the new analytic method by which Mr.~Littlewood and
I have recently studied the problem.

Landau's theorem suggests the introduction of another
function of~$k$, which I will call~$G(k)$, of the same general
character as~$g(k)$, but I think probably more fundamental.
This number~$G(k)$ is defined as being the least number for
which it is true that
%% -----File: 023.png---
\begin{quotation}
\textit{`every \DPtypo{member}{number} \textsc{from a certain point onwards} is the
sum of~$G(k)$ $k$-th~powers or less.'}
\end{quotation}
It is obvious that the existence of~$g(k)$ involves that of~$G(k)$,
and that $G(k) \leq g(k)$. When $k = 2$, both numbers
are~$4$; but $G(3) \leq 8$, by Landau's theorem, while $g(3) = 9$;
and doubtless $G(k) < g(k)$ in general. It is important also
to observe that, conversely, the existence of~$G(k)$ involves
that of~$g(k)$. For, if $G(k)$~exists, all numbers beyond a
certain limit~$\gamma$ are sums of~$G(k)$ $k$-th~powers or less. But
all numbers less than~$\gamma$ are sums of $\gamma$~ones or less, and therefore
$g(k)$ certainly cannot exceed the greater of $G(k)$~and~$\gamma$.

I said that $G(k)$ seemed to me the more fundamental of
these numbers, and it is easy to see why. Let us assume
(as is no doubt true) that the only numbers which require
$9$~cubes for their expression are $23$~and~$239$. This is a very
curious fact which should be interesting to any genuine
arithmetician; for it ought to be true of an arithmetician
that, as has been said of Mr.~Ramanujan, and in his case
at any rate with absolute truth, that `every positive integer
is one of his personal friends'. But it would be absurd to
pretend that it is one of the profounder truths of higher
arithmetic: it is nothing more than an entertaining arithmetical
fluke. It is Landau's~$8$ and not Wieferich's~$9$ that
is the profoundly interesting number.

The real value of~$G(3)$ is still unknown. It cannot be
less than~$4$; for every number is congruent to~$0$, or~$1$, or~$-1$
to modulus~$3$, and it is an elementary deduction that
every cube is congruent to~$0$, or~$1$, or~$-1$ to modulus~$9$.
From this it follows that the sum of three cubes cannot be
of the form $9m + 4$~or $9m + 5$: for such numbers at least
$4$~cubes are necessary, so that $G(3) \geq 4$. But whether
$G(3)$~is $4$,~$5$, $6$, $7$, or~$8$ is one of the deepest mysteries
of arithmetic.

It is worth while to glance at the evidence of computation.
%% -----File: 024.png---
Dase, at the instance of Jacobi, tabulated the minimum
number of cubes for values of~$n$ from~$1$ to~$12,000$, and Daublensky
von~Sterneck has extended the table to~$40,000$.
Some of the results are shown in \TableRef{Table~II}{II.}\@.
\begin{table}[hbt!]
\Table{II.}
\[
\begin{array}{rrrrrrrrrc}
& \ColHead{1}& \ColHead{2} & \ColHead{3} & \ColHead{4} & \ColHead{5} & \ColHead{6} & \ColHead{7} & \ColHead{8} & \ColHead{9} \\
    1\EnDash\phantom{0}1000& 10& 41& 122& 242& 293& 202&      73&       15&        2\\
 1000\EnDash\phantom{0}2000&  2& 27& 113& 283& 358& 194&      23&  \EmDash&  \EmDash\\
 9000\EnDash10000&  1& 17& 121& 377& 401&  83& \EmDash&  \EmDash&  \EmDash\\
19000\EnDash20000&  1& 12& 100& 400& 426&  61& \EmDash&  \EmDash&  \EmDash\\
29000\EnDash30000&  1& 11& 105& 448& 388&  47& \EmDash&  \EmDash&  \EmDash\\
39000\EnDash40000&  1& 13& 117& 457& 384&  28& \EmDash&  \EmDash&  \EmDash
\end{array}
\]
\end{table}
In each row I have shown a typical thousand numbers,
classified according to the minimum number of cubes by
which they can be expressed. There are $15$~numbers only
for which $8$~are needed, the largest being~$454$. There are
$121$ for which $7$~are needed, the two largest being $5818$ and
$8042$; the distribution of these $121$~numbers in the first
$9$~thousands is
\[
  73,\thickspace 23,\thickspace 7,\thickspace 6,\thickspace 7,\thickspace 4,\thickspace 0,\thickspace 0,\thickspace 1.
\]
If empirical evidence means anything, it seems clear that
$G(3) \leq 6$. I am sure that Professor Townsend and Professor
Lindemann have made countless generalizations on evidence
far less substantial.

It is also clear that, throughout von~Sterneck's tables,
there is a fairly steady, though latterly very slow, decrease
in the proportion of numbers for which even $6$~cubes are
required; but that the table is not sufficiently extensive to
give any very decisive indication as to whether these numbers
disappear or not. It seemed to me this was a case in
which further evidence would be worth having. To calculate
a \emph{systematic} table on the scale required would be a work
of years. It is possible, however, to obtain some indication
%% -----File: 025.png---
of the probable truth, without any superhuman patience,
by exploring a selected stratum of much larger numbers.
Dr.~Ruckle of G�ttingen recently undertook this task at
my request, and I am glad to be able to tell you his
results. He found, for the $2,000$~numbers immediately
below $1,000,000$, the following distribution.
\[
\begin{array}{lccccccc}
 & 1 & 2 & 3 & 4 & 5 & 6 & 7 \\
998000\EnDash  999000 & 0 & 1 & 98 & 640 & 262 & 1 & 0\\%[**TN: there is an error in this row, as it adds up to 1002 not 1000]
999000\EnDash 1000000 & 1 & 1 & 94 & 614 & 289 & 1 & 0\\
\end{array}
\]
You will observe that the $6$-cube numbers have all but
disappeared, and that there is a quite marked turnover
from $5$ to~$4$. Conjecture in such a matter is extremely
rash, but I am on the whole disposed to predict with some
confidence that $G(3) \leq 5$. If I were asked to choose between
$5$~and~$4$, all I could say would be this. That $G(3)$ should
be~$4$ would harmonize admirably, so far as we can see
at present, with the general trend of Mr.~Littlewood's and
my researches. But it is plain that, if the $5$-cube numbers
too do ultimately disappear, it can only be among numbers
the writing of which would tax the resources of the decimal
notation; and at present we cannot \emph{prove} even that $G(3) \leq 7$,
though here success seems not impossible.

With the fourth powers or biquadrates we have been
very much more successful. I have explained that $g(4)$~lies
between $19$~and~$37$. As regards~$G(4)$, we have here no
numerical evidence on the same scale as for cubes. Any
fourth power is congruent to $0$~or~$1$ to modulus~$16$, and
from this it follows that no number congruent to~$15$ to
modulus~$16$ can be the sum of less than~$15$ fourth powers.
Thus $G(4) \geq 15$; and Kempner, by a slight elaboration of
this simple argument, has proved that $G(4) \geq 16$. No
better upper bound was known before than the~$37$ of
Wieferich, but here Mr.~Littlewood and I have been able
%% -----File: 026.png---
to make a very substantial improvement, first to~$33$ and
finally to~$21$. Thus $G(4)$~lies between $16$~and~$21$, and
the margin is comparatively small.

I turn now to the general case. In the years up to
1909, the existence proof was effected, and upper bounds
for~$g(k)$ determined, for the values of~$k$ from~$2$ to~$8$ inclusive
and for $k = 10$. These upper bounds are shown
in the first row of \TableRef{Table~III}{III.}; that for~$10$, which is not
included, is somewhere in the neighbourhood of~$140,000$.
\begin{table}[hbt!]
\Table{III.}
\[
\begin{array}{@{}l@{\ }cccrrrr}
& 2&    3&    4& \ColHead{5}& \ColHead{6}& \ColHead{7}& \ColHead{8}\\
g(k) \leq&
  4&    9&   37&   58&   478&   3806&  31353\\
\DPtypo{G(k)}{g(k)} \geq \bigl[\left(\frac{3}{2}\right)^{k}\bigr] + 2^{k} - 2 =&
  4&    9&   19&   37&    73&    143&      279\\
G(k) \leq&
   4& [8]&   37&   58&   478&   3806&  31353\\
G(k) \leq (k - 2) 2^{k-1} + 5 =&
 (5)& (9)&   \mathbf{21}&   \mathbf{53}&   \mathbf{133}&    \mathbf{325}&    \mathbf{773}\\
G(k) \geq k + 1, 4k&
   4&   4&   16&    6&     7&      8&     32
\end{array}
\]
\end{table}
In the second row I have shown the best known lower
bounds, which are given by the simple general formula
which stands to the left, in which $\left[\left(\frac{3}{2}\right)^{k}\right]$~denotes the
integral part of~$\left(\frac{3}{2} \right)^{k}$. It is easily verified, in fact, that
the number
\[
\left(\left[\left(\tfrac{3}{2}\right)^{k}\right] - 1\right) 2^{k} + 2^{k} - 1,
\]
which is less than~$3^{k}$, requires the number of $k$-th~powers
stated.\footnote
  {This observation was made by Bretschneider in~1853.}
It will be observed that the first three numbers
are those which occur in Waring's enunciation.

Waring's problem, as I have defined it---the problem,
that is to say, of finding a general existence proof
for~$g(k)$, and \textit{a~fortiori} for~$G(k)$---was ultimately solved
by Hilbert, once more in~1909. I wish that I had time
to give a proper account of his justly famous memoir,
which raised the whole discussion at once on to an
%% -----File: 027.png---
altogether higher level. As it is, I must confine myself
to one or two extremely inadequate remarks. The
proof falls into two parts. The first part is what I may
call semi-transcendental. It is not fully transcendental in
the sense in which, for example, the classical proofs in the
theory of the distribution of primes are transcendental, for
it does not make use of the machinery of the theory of
analytic functions of a complex variable; but it uses the
methods of the integral calculus, and is therefore not
fully elementary. Hilbert set out with what would appear
at first sight to be the singularly ill-adapted weapon of
a volume integral in space of $25$~dimensions, a number
which he was afterwards able to reduce to~$5$. The formula
which he ultimately used is
\begin{multline*}
(x_{1}^{2} + x_{2}^{2} + x_{3}^{2} + x_{4}^{2} + x_{5}^{2})^{k}\\
=  C\iiiiint (x_{1}t_{1} + x_{2}t_{2} + x_{3}t_{3} + x_{4}t_{4} +
             x_{5}t_{5})^{2k}\, dt_{1} \dots dt_{5},
\end{multline*}
where $C$~is a certain constant, viz.\
\[
\frac{(2k + 1)(2k + 3)(2k + 5)}{8\pi^{2}},
\]
and the integration is effected over the interior of the
hypersphere
\[
t_{1}^{2} + t_{2}^{2} + t_{3}^{2} + t_{4}^{2} + t_{5}^{2} = 1.
\]

Starting from this formula he was able, by an exceedingly
ingenious process based upon the definition of a
definite integral as the limit of a finite sum, to prove
the existence in the general case of algebraical identities
analogous to that used by Liouville and his followers
when $k$~is~$4$. It should be observed that Hilbert's
proof is essentially an \emph{existence proof}; his method is
not effective for the actual determination of these identities
%% -----File: 028.png---
even in the simplest cases. The identities which are known
for special values of~$k$ have been obtained by common
algebra, and are, after the first few values of~$k$, excessively
complicated. The simplest known identity for $k = 10$, for
instance, is
\begin{multline*}
22680(x_{1}^{2} + x_{2}^{2} + x_{3}^{2} + x_{4}^{2})^{5}\\
\begin{aligned}
&= 9\sum^{(8)} (x_{1} � x_{2} � x_{3} � x_{4})^{10}
  + \sum^{(48)} (2x_{1} � x_{2} � x_{3})^{10}\\
&+ 180\sum^{(12)} (x_{1} � x_{2})^{10}
  + 9\sum^{(4)} (2x_{1})^{10},
\end{aligned}
\end{multline*}
where the figures in brackets show the number of terms
under the signs of summation. However, the identities
exist; and it should be clear to you, after our discussion of
the case $k = 4$, that they enable us at once to obtain a
proof in succession for $k = 2$,~$4$, $8$, $16$,~$\dots$ and generally
whenever $k$~is a power of~$2$. This concludes the first and
most characteristic part of Hilbert's argument. The second
part, in which the conclusion is extended to every value of~$k$,
is purely algebraical.

Hilbert's work has been reconsidered and simplified by
a number of writers, most notably by Dr.~Stridsberg of
Stockholm, and the ultimate result of their work has been
to eliminate the transcendental elements from the proof
entirely. The proof, as they have left it, is fully elementary;
it does not involve any reference to integrals, or to
any kind of limiting process, but depends simply on an
ingenious system of equations derived by the processes of
finite algebra. It remains a pure existence proof, and
throws no light on the value of~$g(k)$.

It would hardly be possible for me to exaggerate the
admiration which I feel for the solution of this historic
problem of which I have been compelled to give so bald
%% -----File: 029.png---
and summary a description. Within the limits which it
has set for itself, it is absolutely and triumphantly successful,
and it stands with the work of Hadamard and de~la
Vall�e-Poussin, in the theory of primes, as one of the landmarks
in the modern history of the theory of numbers.
But there is an enormous amount which remains to be
done, and it would seem that, if we are to interpret
Waring's problem in the widest possible sense, if we are to
get into real contact with the actual values of our numbers
$g(k)$~and~$G(k)$, still more if we are to attack all the obvious
problems connected with the number of representations,
then essentially different and inherently more powerful
methods are required. There is one armoury only in which
such more powerful weapons can be found, that of the
modern theory of functions. In short we must learn how
to apply Cauchy's Theorem to the problem, and that is
what Mr.~Littlewood and I have set out to do.

The first step is fairly obvious. The formulae are
slightly simpler when $k$~is \emph{even}. The number of representations
of~$n$ as the sum of~$s$ $k$\DPtypo{}{-}th~powers, which we may
denote in general by
\[
r_{k, s}(n),
\]
is then the coefficient of~$x^{n}$ in the generating function
\[
1 + \sum_{1}^{\infty} r_{k, s}(n)x^{n} = \left(f(x)\right)^{s},
\]
where
\[
f(x) = 1 + 2x^{1^{k}} + 2x^{2^{k}} + 2x^{3^{k}} + \dots\DPtypo{}{.}
\]
This formula involves certain conventions as to the order
and sign of the numbers which occur in the representations
which are to be reckoned as distinct; but the complications
so introduced are trivial and I need not dwell on them.
The series is convergent when $|x| < 1$, and, by Cauchy's
%% -----File: 030.png---
Theorem, we have
\[
r_{k,s}(n) = \frac{1}{2\pi i} \int \frac{\left(f(x)\right)^{s}}{x^{n+1}}\, dx,
\]
the path of integration being a circle whose centre is at
the origin and whose radius is less than unity.

All this is simple enough; but the further study of the
integral is very intricate and difficult, and I cannot attempt
to do more than to give a rough idea of the obstacles that
have to be surmounted. Let us contrast the integral for a
moment with that which would stand in its place in the
`trivial' problem to which I referred early in my lecture.
There the subject of integration would be a \emph{rational} function,
with a finite number of poles all situated on the unit
circle. We could deform the contour into one which lies
wholly at a considerable distance from the origin and in
which, owing to the factor~$x^{n+1}$ in the denominator, every
element is very small when $n$~is large. We should have,
of course, to introduce corrections corresponding to the
residues at the poles; and it is just these corrections which
would give the dominant terms of an approximate formula
by means of which our coefficients could be studied. In
the present case we have no such simple recourse; for
every point of the unit circle is a singularity of an exceedingly
complicated kind, and the circle as a whole is a
barrier across which it is impossible to deform the contour.
It is of course for this reason that no successful application
of the method has been made before.

Our fundamental idea for overcoming the difficulty is
as follows. Among the continuous mass of singularities
which covers up the circle, it is possible to pick out a class
which to a certain extent dominates the rest. These special
singularities are those associated with the \emph{rational} points
of the circle, that is to say, the points
\[
x = e^{2p\pi i/q},
\]
%% -----File: 031.png---
where $p/q$~is a rational fraction in its lowest terms. This
class of points is indeed an \emph{infinite} class; but the infinity
is, in Cantor's phrase, only an \emph{enumerable} infinity; and
the points can therefore be arranged in a simply infinite
series, on the model of the series
\[
\tfrac{0}{1},\thickspace \tfrac{1}{2},\thickspace \tfrac{1}{3},\thickspace \tfrac{2}{3},\thickspace \tfrac{1}{4},\thickspace
\tfrac{3}{4},\thickspace \tfrac{1}{5},\thickspace \tfrac{2}{5},\thickspace \tfrac{3}{5},\thickspace \tfrac{4}{5},\thickspace
\tfrac{1}{6},\thickspace \tfrac{5}{6},\thickspace \tfrac{1}{7},\thickspace \dots.
\]
In the neighbourhood of these points the behaviour of the
function is\DPtypo{,}{} sufficiently complex indeed, but simpler than
elsewhere. The function has, to put the matter in a rough
and popular way, a general tendency to become large in
the neighbourhood of the unit circle, but this tendency is
most pronounced near these particular points. They are
not only the \emph{simplest} but also the \emph{heaviest} singularities;
their weight is greatest when the denominator~$q$ is smallest,
decreases as $q$~increases, and (as a physicist would say)
becomes infinitely small when $q$~is infinitely large. There
is, therefore, at any rate, the hope that we may be able to
isolate the contributions of each of these selected points,
and obtain, by adding them together, a series which may
give a genuine approximation to our coefficient.

I owe to Professor Harald Bohr of Copenhagen a
picturesque illustration which may help to elucidate the
general nature of our argument. Imagine the unit circle
as a thin circular rail, to which are attached an infinite
number of small lights of varying intensity, each illuminating
a certain angle immediately in front of it. The
brightest light is at $x = 1$, corresponding to $p = 0$, $q = 1$;
the next brightest at $x = -1$, corresponding to $p = 1$,
$q = 2$; the next at $x = e^{2\pi i/3}$ and $e^{4\pi i/3}$, and so on. We
have to arrange the inner circle, the circle of integration,
in the position of maximum illumination. If it is too far
away the light will not reach it; if too near, the arcs
which fall within the angles of illumination will be too
%% -----File: 032.png---
narrow, and the light will not cover it completely. Is it
possible to place it where it will receive a satisfactorily
uniform illumination?

The answer is that this is \emph{only} possible when $k$~is~$2$.
Our functions are then elliptic functions; the lights are the
formulae of the theory of linear transformation; and we
can find a position of the inner circle in which it falls
entirely under their rays. We are thus led to a solution of
the problem of the squares which is in all essential respects
complete. But when $k$~exceeds~$2$ the result is less satisfactory.
The angle of the lights is then too narrow; the
beams which they emit, instead of spreading out with
reasonable regularity, are shaped like torpedoes or cigars;
however we move our circle a part remains in darkness.
It would seem that this difficulty, which held up our
researches for something like two years, is the really
characteristic difficulty of the general problem. It cannot
be solved until we have found some other source of light.

It was only after the most prolonged and painful efforts
that we were able to discover such another source. It is
possible not only to hang lights upon the rail, but also, to
a certain extent, to cause the rail itself to glow. The
illumination which can be induced in this manner is irritatingly
faint, and it is for this reason that our results are
not yet all that we desire; but it is enough to make the
dark places dimly visible and to enable us to prove a great
deal more than has been proved before.

The actual results which we obtain are these. We find
that there is a certain series, which we call the \emph{singular
series}, which is plainly the key to the solution. This
series is
\[
\S = \sum \left(\frac{S_{p,q}}{q}\right)^{s} e^{2np\pi i/q},
\]
%% -----File: 033.png---
where
\[
S_{p,q} = \sum_{h=0}^{q-1} e^{2h^{k}p\pi i/q}
\]
---a sum which reduces, when $k = 2$, to one of what are
known as `Gauss's sums'---and the summation extends,
first to all values of $p$ less than and prime to~$q$, and
secondly to all positive integral values of~$q$. The genesis
of the series is this. We associate with the rational point
$x = e^{2p\pi i/q}$ an auxiliary power series
\[
f_{p,q}(x)= \sum_{n} c_{p,q,n}x^{n},
\]
which (\textit{a})~is as simple and natural as we can make it, and
(\textit{b})~behaves perfectly regularly at all points of the unit circle
except at the one point with which we are particularly concerned.
We then add together all these auxiliary functions,
and endeavour to approximate to the coefficient of our original
series by summing the auxiliary coefficients over all values
of $p$~and~$q$. The process is, at bottom, one of `decomposition
into simple elements', applied in an unusual way.

Our final formula for the number of representations is
\[
r_{k,s}(n) =
  \frac{\left\{2\Gamma\Bigl(1 + \dfrac{1}{k}\Bigr)\right\}^{s}}
       {\Gamma\Bigl(1 + \dfrac{s}{k}\Bigr)}
        n^{\frac{s}{k}-1} \S + O(n^{\sigma}),
\]
the second term denoting an error less than a constant
multiple of~$n^{\sigma}$, and $\sigma$~being a number which is less than
$\dfrac{s}{k} - 1$ at any rate for sufficiently large values of~$s$. The
second term is then of lower order than the first. Further,
the first term is real, and it may be shown, if $s$~surpasses
a certain limit, to be \emph{positive}. If both these conditions
are satisfied, and $n$~is sufficiently large, then $r_{k,s}(n)$
%% -----File: 034.png---
cannot be zero, and representations of~$n$ by $s$ $k$\DPtypo{}{-}th~powers
certainly exist. The way is thus open to a proof of the
existence of~$G(k)$; if $G(k)$~exists, so also does~$g(k)$, and
Waring's problem is solved.

The structure of the dominant term in our general
formula is best realized by considering some special cases.
In \TableRef{Table~IV}{IV.} I have written out the leading terms of~$\S$,
\begin{table}[hbt!]
\Table{IV.}
\[
\begin{array}{@{}l}
\ColHead{k = 2.}\\
\S = 1 + 0 +
  \dfrac{2}{3^{\frac{1}{2}s}}
    \cos\left(\dfrac{2}{3} n\pi - \dfrac{1}{2} s\pi\right) +
  \dfrac{2^{\frac{1}{2}s+1}}{4^{\frac{1}{2}s}}
    \cos\left(\dfrac{1}{2}n\pi - \dfrac{1}{4} s\pi\right)\\[3ex]
  \hspace*{3cm}+ \dfrac{2}{5^{\frac{1}{2} s}}
        \left\{\cos\dfrac{2}{5} n\pi +
                 \cos\left(\dfrac{4}{5} n\pi - s\pi\right)\right\} +
                   0 + \dots.\\[4ex]
\ColHead{k = 3, s = 7.}\\[1ex]
\S = 1 + 0.610 \cos \frac{2}{9} n\pi + 0.130 \cos \frac{2}{7} n\pi +
         0.078 \cos \frac{6}{7} n\pi + \dots.\\[3ex]
\ColHead{k = 4, s = 33.}\\[1ex]
\S = 1 + 1.054 \cos (\frac{1}{8} n\pi - \frac{1}{16} \pi) +
       0.147 \cos (\frac{1}{4} n\pi - \frac{1}{8} \pi) + \dots.\\[3ex]
\ColHead{k = 4, s = 21.}\\[1ex]
\S = 1 + 1.331 \cos (\frac{1}{8} n\pi + \frac{11}{16} \pi) +
           0.379 \cos (\frac{1}{4} n\pi - \frac{5}{8} \pi) + \dots.
\end{array}
\]
\end{table}
first when $k = 2$ and $s$~is arbitrary, and then for $7$~cubes
and for $33$~and $21$~biquadrates. There are certain
characteristics common to all these series. The terms
diminish rapidly; in each case only a very few are of real
importance: and they are oscillatory, with a period which
increases as the amplitude of the oscillations decreases.
The series for the cubes is easily shown to be positive;
but we cannot deduce that $r_{3,7}(n)$ is positive, and draw
consequences as to the representation of numbers by $7$~cubes,
%% -----File: 035.png---
because in this case we cannot dispose satisfactorily of the
error term~$O(n^{\sigma})$ in the general formula. In the two
cases relating to fourth powers which I have chosen, the
discussion of the series itself is rather more delicate, for
there is in each of them one term which can be negative
and greater than~$1$. But the discussion can be brought to
a satisfactory conclusion, and, as in this case we are able
to prove that the error term is really of lower order, we
obtain what we desire. \emph{Every large number is the sum of
$21$~fourth powers or less}; $G(4) \leq 21$. Further, we have
obtained a genuine asymptotic formula for the number
of representations, which can be used for the study of the
representations of numbers of particular forms. We can
show, for example, that a large number of the form
$16n + 10$ can be expressed by $21$~biquadrates in about $200$~times
more ways than one of the form~$16n + 2$.

If the method of which I have tried to give some general
idea is compared with those which have previously been
applied to the problem, it will be found that it has three
very great advantages. In the first place it is inherently
very much more powerful. It brings us for the first time
into relation with the series on which the solution in the
last resort depends, and tells us, approximately but truly,
what the number of representations really is. Secondly, it
gives us numerical results which, as soon as $k$~exceeds~$3$, are
far in advance of any known before. These numbers
are those in the fourth row of \TableRef{Table~III}{III.}\@.\footnote
  {The thick type indicates a new result. The (5)~and~(9) in round
  brackets are inferior to results already known. Our method is easily
  adapted to deal with the case $k = 2$ completely; but it will not at
  present yield Landau's~$8$, which is therefore enclosed in square
  brackets.}
It will be seen
that these numbers conform to a simple law, and that is
the third advantage of the method, that it is not a mere
%% -----File: 036.png---
existence proof, but gives us a definite upper bound for~$G(k)$
for all values of~$k$, viz.\
\[
G(k) \leq (k - 2) 2^{k-1} + 5.
\]

In the last row of the table~I have shown all that is known
about~$G(k)$ on the other side. In all cases $G(k) \geq k + 1$,
while if $k$~is a power of~$2$ we can say more, namely that
$G(k) \geq 4k$. A comparison between this row of figures and
that above it is enough to show the room which remains
for further research. It is beyond question that our
numbers are still very much too large; and there is no
sort of finality about our researches, for which the best
that we claim is that they embody a method which opens
the door for more.

I will conclude by one word as to the application of
our method to another and a still more difficult problem.
It was asserted by Goldbach in 1742 that \emph{every even number
is the sum of two odd primes}. Goldbach's assertion
remains unproved; it has not even been proved that every
number~$n$ is the sum of $10$~primes, or of~$100$, or of any number
independent of~$n$. Our method is applicable in principle to
this problem also. We cannot solve the problem, but we
can open the first serious attack upon it, and bring it into
relation with the established prime number theory. The
most which we can accomplish at present is as follows.
We have to assume the truth of the notorious Riemann
hypothesis concerning the zeros of the Zeta-function, and
indeed in a generalized and extended form. If we do this
we can prove, not Goldbach's Theorem indeed, but the next
best theorem of the kind, viz.\ that \emph{every odd number}, at any
rate from a certain point onwards, \emph{is the sum of three odd
primes}. It is an imperfect and provisional result, but it is
the first serious contribution to the solution of the problem.
%% -----File: 037.png---
\BackMatter
\null
\vfill
\subsection*{\centering\normalfont{POSTSCRIPT}}
Srinivasa Ramanujan, F.R.S., Fellow of Trinity
College, Cambridge, died in India on April~26, 1920,
aged~32.

An account of his life and mathematical activities
will be published in Vol.~19 of the \textit{Proceedings of
the London Mathematical Society}.
\vfill
%% -----File: 038.png---
\newpage
\null
\vfill
\begin{center}
PRINTED IN ENGLAND \\
AT THE OXFORD UNIVERSITY PRESS \\
\vfill
\end{center}

%%%%%%%%%%%%%%%%%%%%%%%%% GUTENBERG LICENSE %%%%%%%%%%%%%%%%%%%%%%%%%%
\newpage
\BookMark{0}{PG License.}
\begin{PGtext}
End of the Project Gutenberg EBook of Some Famous Problems of the Theory of
Numbers and in particular Waring's Problem, by G. H. (Godfrey Harold) Hardy

*** END OF THIS PROJECT GUTENBERG EBOOK FAMOUS PROBLEMS OF THEORY OF NUMBERS ***

***** This file should be named 37030-pdf.pdf or 37030-pdf.zip *****
This and all associated files of various formats will be found in:
        http://www.gutenberg.org/3/7/0/3/37030/

Produced by Anna Hall, Brenda Lewis and the Online
Distributed Proofreading Team at http://www.pgdp.net (This
file was produced from images generously made available
by The Internet Archive/American Libraries.)


Updated editions will replace the previous one--the old editions
will be renamed.

Creating the works from public domain print editions means that no
one owns a United States copyright in these works, so the Foundation
(and you!) can copy and distribute it in the United States without
permission and without paying copyright royalties.  Special rules,
set forth in the General Terms of Use part of this license, apply to
copying and distributing Project Gutenberg-tm electronic works to
protect the PROJECT GUTENBERG-tm concept and trademark.  Project
Gutenberg is a registered trademark, and may not be used if you
charge for the eBooks, unless you receive specific permission.  If you
do not charge anything for copies of this eBook, complying with the
rules is very easy.  You may use this eBook for nearly any purpose
such as creation of derivative works, reports, performances and
research.  They may be modified and printed and given away--you may do
practically ANYTHING with public domain eBooks.  Redistribution is
subject to the trademark license, especially commercial
redistribution.



*** START: FULL LICENSE ***

THE FULL PROJECT GUTENBERG LICENSE
PLEASE READ THIS BEFORE YOU DISTRIBUTE OR USE THIS WORK

To protect the Project Gutenberg-tm mission of promoting the free
distribution of electronic works, by using or distributing this work
(or any other work associated in any way with the phrase "Project
Gutenberg"), you agree to comply with all the terms of the Full Project
Gutenberg-tm License (available with this file or online at
http://gutenberg.net/license).


Section 1.  General Terms of Use and Redistributing Project Gutenberg-tm
electronic works

1.A.  By reading or using any part of this Project Gutenberg-tm
electronic work, you indicate that you have read, understand, agree to
and accept all the terms of this license and intellectual property
(trademark/copyright) agreement.  If you do not agree to abide by all
the terms of this agreement, you must cease using and return or destroy
all copies of Project Gutenberg-tm electronic works in your possession.
If you paid a fee for obtaining a copy of or access to a Project
Gutenberg-tm electronic work and you do not agree to be bound by the
terms of this agreement, you may obtain a refund from the person or
entity to whom you paid the fee as set forth in paragraph 1.E.8.

1.B.  "Project Gutenberg" is a registered trademark.  It may only be
used on or associated in any way with an electronic work by people who
agree to be bound by the terms of this agreement.  There are a few
things that you can do with most Project Gutenberg-tm electronic works
even without complying with the full terms of this agreement.  See
paragraph 1.C below.  There are a lot of things you can do with Project
Gutenberg-tm electronic works if you follow the terms of this agreement
and help preserve free future access to Project Gutenberg-tm electronic
works.  See paragraph 1.E below.

1.C.  The Project Gutenberg Literary Archive Foundation ("the Foundation"
or PGLAF), owns a compilation copyright in the collection of Project
Gutenberg-tm electronic works.  Nearly all the individual works in the
collection are in the public domain in the United States.  If an
individual work is in the public domain in the United States and you are
located in the United States, we do not claim a right to prevent you from
copying, distributing, performing, displaying or creating derivative
works based on the work as long as all references to Project Gutenberg
are removed.  Of course, we hope that you will support the Project
Gutenberg-tm mission of promoting free access to electronic works by
freely sharing Project Gutenberg-tm works in compliance with the terms of
this agreement for keeping the Project Gutenberg-tm name associated with
the work.  You can easily comply with the terms of this agreement by
keeping this work in the same format with its attached full Project
Gutenberg-tm License when you share it without charge with others.

1.D.  The copyright laws of the place where you are located also govern
what you can do with this work.  Copyright laws in most countries are in
a constant state of change.  If you are outside the United States, check
the laws of your country in addition to the terms of this agreement
before downloading, copying, displaying, performing, distributing or
creating derivative works based on this work or any other Project
Gutenberg-tm work.  The Foundation makes no representations concerning
the copyright status of any work in any country outside the United
States.

1.E.  Unless you have removed all references to Project Gutenberg:

1.E.1.  The following sentence, with active links to, or other immediate
access to, the full Project Gutenberg-tm License must appear prominently
whenever any copy of a Project Gutenberg-tm work (any work on which the
phrase "Project Gutenberg" appears, or with which the phrase "Project
Gutenberg" is associated) is accessed, displayed, performed, viewed,
copied or distributed:

This eBook is for the use of anyone anywhere at no cost and with
almost no restrictions whatsoever.  You may copy it, give it away or
re-use it under the terms of the Project Gutenberg License included
with this eBook or online at www.gutenberg.net

1.E.2.  If an individual Project Gutenberg-tm electronic work is derived
from the public domain (does not contain a notice indicating that it is
posted with permission of the copyright holder), the work can be copied
and distributed to anyone in the United States without paying any fees
or charges.  If you are redistributing or providing access to a work
with the phrase "Project Gutenberg" associated with or appearing on the
work, you must comply either with the requirements of paragraphs 1.E.1
through 1.E.7 or obtain permission for the use of the work and the
Project Gutenberg-tm trademark as set forth in paragraphs 1.E.8 or
1.E.9.

1.E.3.  If an individual Project Gutenberg-tm electronic work is posted
with the permission of the copyright holder, your use and distribution
must comply with both paragraphs 1.E.1 through 1.E.7 and any additional
terms imposed by the copyright holder.  Additional terms will be linked
to the Project Gutenberg-tm License for all works posted with the
permission of the copyright holder found at the beginning of this work.

1.E.4.  Do not unlink or detach or remove the full Project Gutenberg-tm
License terms from this work, or any files containing a part of this
work or any other work associated with Project Gutenberg-tm.

1.E.5.  Do not copy, display, perform, distribute or redistribute this
electronic work, or any part of this electronic work, without
prominently displaying the sentence set forth in paragraph 1.E.1 with
active links or immediate access to the full terms of the Project
Gutenberg-tm License.

1.E.6.  You may convert to and distribute this work in any binary,
compressed, marked up, nonproprietary or proprietary form, including any
word processing or hypertext form.  However, if you provide access to or
distribute copies of a Project Gutenberg-tm work in a format other than
"Plain Vanilla ASCII" or other format used in the official version
posted on the official Project Gutenberg-tm web site (www.gutenberg.net),
you must, at no additional cost, fee or expense to the user, provide a
copy, a means of exporting a copy, or a means of obtaining a copy upon
request, of the work in its original "Plain Vanilla ASCII" or other
form.  Any alternate format must include the full Project Gutenberg-tm
License as specified in paragraph 1.E.1.

1.E.7.  Do not charge a fee for access to, viewing, displaying,
performing, copying or distributing any Project Gutenberg-tm works
unless you comply with paragraph 1.E.8 or 1.E.9.

1.E.8.  You may charge a reasonable fee for copies of or providing
access to or distributing Project Gutenberg-tm electronic works provided
that

- You pay a royalty fee of 20% of the gross profits you derive from
     the use of Project Gutenberg-tm works calculated using the method
     you already use to calculate your applicable taxes.  The fee is
     owed to the owner of the Project Gutenberg-tm trademark, but he
     has agreed to donate royalties under this paragraph to the
     Project Gutenberg Literary Archive Foundation.  Royalty payments
     must be paid within 60 days following each date on which you
     prepare (or are legally required to prepare) your periodic tax
     returns.  Royalty payments should be clearly marked as such and
     sent to the Project Gutenberg Literary Archive Foundation at the
     address specified in Section 4, "Information about donations to
     the Project Gutenberg Literary Archive Foundation."

- You provide a full refund of any money paid by a user who notifies
     you in writing (or by e-mail) within 30 days of receipt that s/he
     does not agree to the terms of the full Project Gutenberg-tm
     License.  You must require such a user to return or
     destroy all copies of the works possessed in a physical medium
     and discontinue all use of and all access to other copies of
     Project Gutenberg-tm works.

- You provide, in accordance with paragraph 1.F.3, a full refund of any
     money paid for a work or a replacement copy, if a defect in the
     electronic work is discovered and reported to you within 90 days
     of receipt of the work.

- You comply with all other terms of this agreement for free
     distribution of Project Gutenberg-tm works.

1.E.9.  If you wish to charge a fee or distribute a Project Gutenberg-tm
electronic work or group of works on different terms than are set
forth in this agreement, you must obtain permission in writing from
both the Project Gutenberg Literary Archive Foundation and Michael
Hart, the owner of the Project Gutenberg-tm trademark.  Contact the
Foundation as set forth in Section 3 below.

1.F.

1.F.1.  Project Gutenberg volunteers and employees expend considerable
effort to identify, do copyright research on, transcribe and proofread
public domain works in creating the Project Gutenberg-tm
collection.  Despite these efforts, Project Gutenberg-tm electronic
works, and the medium on which they may be stored, may contain
"Defects," such as, but not limited to, incomplete, inaccurate or
corrupt data, transcription errors, a copyright or other intellectual
property infringement, a defective or damaged disk or other medium, a
computer virus, or computer codes that damage or cannot be read by
your equipment.

1.F.2.  LIMITED WARRANTY, DISCLAIMER OF DAMAGES - Except for the "Right
of Replacement or Refund" described in paragraph 1.F.3, the Project
Gutenberg Literary Archive Foundation, the owner of the Project
Gutenberg-tm trademark, and any other party distributing a Project
Gutenberg-tm electronic work under this agreement, disclaim all
liability to you for damages, costs and expenses, including legal
fees.  YOU AGREE THAT YOU HAVE NO REMEDIES FOR NEGLIGENCE, STRICT
LIABILITY, BREACH OF WARRANTY OR BREACH OF CONTRACT EXCEPT THOSE
PROVIDED IN PARAGRAPH 1.F.3.  YOU AGREE THAT THE FOUNDATION, THE
TRADEMARK OWNER, AND ANY DISTRIBUTOR UNDER THIS AGREEMENT WILL NOT BE
LIABLE TO YOU FOR ACTUAL, DIRECT, INDIRECT, CONSEQUENTIAL, PUNITIVE OR
INCIDENTAL DAMAGES EVEN IF YOU GIVE NOTICE OF THE POSSIBILITY OF SUCH
DAMAGE.

1.F.3.  LIMITED RIGHT OF REPLACEMENT OR REFUND - If you discover a
defect in this electronic work within 90 days of receiving it, you can
receive a refund of the money (if any) you paid for it by sending a
written explanation to the person you received the work from.  If you
received the work on a physical medium, you must return the medium with
your written explanation.  The person or entity that provided you with
the defective work may elect to provide a replacement copy in lieu of a
refund.  If you received the work electronically, the person or entity
providing it to you may choose to give you a second opportunity to
receive the work electronically in lieu of a refund.  If the second copy
is also defective, you may demand a refund in writing without further
opportunities to fix the problem.

1.F.4.  Except for the limited right of replacement or refund set forth
in paragraph 1.F.3, this work is provided to you 'AS-IS' WITH NO OTHER
WARRANTIES OF ANY KIND, EXPRESS OR IMPLIED, INCLUDING BUT NOT LIMITED TO
WARRANTIES OF MERCHANTIBILITY OR FITNESS FOR ANY PURPOSE.

1.F.5.  Some states do not allow disclaimers of certain implied
warranties or the exclusion or limitation of certain types of damages.
If any disclaimer or limitation set forth in this agreement violates the
law of the state applicable to this agreement, the agreement shall be
interpreted to make the maximum disclaimer or limitation permitted by
the applicable state law.  The invalidity or unenforceability of any
provision of this agreement shall not void the remaining provisions.

1.F.6.  INDEMNITY - You agree to indemnify and hold the Foundation, the
trademark owner, any agent or employee of the Foundation, anyone
providing copies of Project Gutenberg-tm electronic works in accordance
with this agreement, and any volunteers associated with the production,
promotion and distribution of Project Gutenberg-tm electronic works,
harmless from all liability, costs and expenses, including legal fees,
that arise directly or indirectly from any of the following which you do
or cause to occur: (a) distribution of this or any Project Gutenberg-tm
work, (b) alteration, modification, or additions or deletions to any
Project Gutenberg-tm work, and (c) any Defect you cause.


Section  2.  Information about the Mission of Project Gutenberg-tm

Project Gutenberg-tm is synonymous with the free distribution of
electronic works in formats readable by the widest variety of computers
including obsolete, old, middle-aged and new computers.  It exists
because of the efforts of hundreds of volunteers and donations from
people in all walks of life.

Volunteers and financial support to provide volunteers with the
assistance they need are critical to reaching Project Gutenberg-tm's
goals and ensuring that the Project Gutenberg-tm collection will
remain freely available for generations to come.  In 2001, the Project
Gutenberg Literary Archive Foundation was created to provide a secure
and permanent future for Project Gutenberg-tm and future generations.
To learn more about the Project Gutenberg Literary Archive Foundation
and how your efforts and donations can help, see Sections 3 and 4
and the Foundation web page at http://www.pglaf.org.


Section 3.  Information about the Project Gutenberg Literary Archive
Foundation

The Project Gutenberg Literary Archive Foundation is a non profit
501(c)(3) educational corporation organized under the laws of the
state of Mississippi and granted tax exempt status by the Internal
Revenue Service.  The Foundation's EIN or federal tax identification
number is 64-6221541.  Its 501(c)(3) letter is posted at
http://pglaf.org/fundraising.  Contributions to the Project Gutenberg
Literary Archive Foundation are tax deductible to the full extent
permitted by U.S. federal laws and your state's laws.

The Foundation's principal office is located at 4557 Melan Dr. S.
Fairbanks, AK, 99712., but its volunteers and employees are scattered
throughout numerous locations.  Its business office is located at
809 North 1500 West, Salt Lake City, UT 84116, (801) 596-1887, email
business@pglaf.org.  Email contact links and up to date contact
information can be found at the Foundation's web site and official
page at http://pglaf.org

For additional contact information:
     Dr. Gregory B. Newby
     Chief Executive and Director
     gbnewby@pglaf.org


Section 4.  Information about Donations to the Project Gutenberg
Literary Archive Foundation

Project Gutenberg-tm depends upon and cannot survive without wide
spread public support and donations to carry out its mission of
increasing the number of public domain and licensed works that can be
freely distributed in machine readable form accessible by the widest
array of equipment including outdated equipment.  Many small donations
($1 to $5,000) are particularly important to maintaining tax exempt
status with the IRS.

The Foundation is committed to complying with the laws regulating
charities and charitable donations in all 50 states of the United
States.  Compliance requirements are not uniform and it takes a
considerable effort, much paperwork and many fees to meet and keep up
with these requirements.  We do not solicit donations in locations
where we have not received written confirmation of compliance.  To
SEND DONATIONS or determine the status of compliance for any
particular state visit http://pglaf.org

While we cannot and do not solicit contributions from states where we
have not met the solicitation requirements, we know of no prohibition
against accepting unsolicited donations from donors in such states who
approach us with offers to donate.

International donations are gratefully accepted, but we cannot make
any statements concerning tax treatment of donations received from
outside the United States.  U.S. laws alone swamp our small staff.

Please check the Project Gutenberg Web pages for current donation
methods and addresses.  Donations are accepted in a number of other
ways including including checks, online payments and credit card
donations.  To donate, please visit: http://pglaf.org/donate


Section 5.  General Information About Project Gutenberg-tm electronic
works.

Professor Michael S. Hart is the originator of the Project Gutenberg-tm
concept of a library of electronic works that could be freely shared
with anyone.  For thirty years, he produced and distributed Project
Gutenberg-tm eBooks with only a loose network of volunteer support.


Project Gutenberg-tm eBooks are often created from several printed
editions, all of which are confirmed as Public Domain in the U.S.
unless a copyright notice is included.  Thus, we do not necessarily
keep eBooks in compliance with any particular paper edition.


Most people start at our Web site which has the main PG search facility:

     http://www.gutenberg.net

This Web site includes information about Project Gutenberg-tm,
including how to make donations to the Project Gutenberg Literary
Archive Foundation, how to help produce our new eBooks, and how to
subscribe to our email newsletter to hear about new eBooks.
\end{PGtext}
% %%%%%%%%%%%%%%%%%%%%%%%%%%%%%%%%%%%%%%%%%%%%%%%%%%%%%%%%%%%%%%%%%%%%%%% %
%                                                                         %
% End of the Project Gutenberg EBook of Some Famous Problems of the Theory of
% Numbers and in particular Waring's Problem, by G. H. (Godfrey Harold) Hardy
%                                                                         %
% *** END OF THIS PROJECT GUTENBERG EBOOK FAMOUS PROBLEMS OF THEORY OF NUMBERS ***
%                                                                         %
% ***** This file should be named 37030-t.tex or 37030-t.zip *****        %
% This and all associated files of various formats will be found in:      %
%         http://www.gutenberg.org/3/7/0/3/37030/                         %
%                                                                         %
% %%%%%%%%%%%%%%%%%%%%%%%%%%%%%%%%%%%%%%%%%%%%%%%%%%%%%%%%%%%%%%%%%%%%%%% %

\end{document}
###
@ControlwordArguments = (
  ['\\BookMark', 1, 0, '', '', 1, 0, '', ''],
  ['\\Title', 1, 1, '', ''],
  ['\\Table', 1, 1, 'Table ', ''],
  ['\\TableRef', 1, 1, '', '', 1, 0, '', ''],
  ['\\DPtypo', 1, 0, '', '', 1, 1, '', ''],
  ['\\First', 1, 1, '', '']
  );
###
This is pdfTeXk, Version 3.141592-1.40.3 (Web2C 7.5.6) (format=pdflatex 2010.5.6)  10 AUG 2011 13:26
entering extended mode
 %&-line parsing enabled.
**37030-t.tex
(./37030-t.tex
LaTeX2e <2005/12/01>
Babel <v3.8h> and hyphenation patterns for english, usenglishmax, dumylang, noh
yphenation, arabic, farsi, croatian, ukrainian, russian, bulgarian, czech, slov
ak, danish, dutch, finnish, basque, french, german, ngerman, ibycus, greek, mon
ogreek, ancientgreek, hungarian, italian, latin, mongolian, norsk, icelandic, i
nterlingua, turkish, coptic, romanian, welsh, serbian, slovenian, estonian, esp
eranto, uppersorbian, indonesian, polish, portuguese, spanish, catalan, galicia
n, swedish, ukenglish, pinyin, loaded.
(/usr/share/texmf-texlive/tex/latex/base/book.cls
Document Class: book 2005/09/16 v1.4f Standard LaTeX document class
(/usr/share/texmf-texlive/tex/latex/base/bk12.clo
File: bk12.clo 2005/09/16 v1.4f Standard LaTeX file (size option)
)
\c@part=\count79
\c@chapter=\count80
\c@section=\count81
\c@subsection=\count82
\c@subsubsection=\count83
\c@paragraph=\count84
\c@subparagraph=\count85
\c@figure=\count86
\c@table=\count87
\abovecaptionskip=\skip41
\belowcaptionskip=\skip42
\bibindent=\dimen102
)

LaTeX Warning: You have requested, on input line 87, version
               `2007/10/19' of document class book,
               but only version
               `2005/09/16 v1.4f Standard LaTeX document class'
               is available.

(/usr/share/texmf-texlive/tex/latex/base/inputenc.sty
Package: inputenc 2006/05/05 v1.1b Input encoding file
\inpenc@prehook=\toks14
\inpenc@posthook=\toks15
(/usr/share/texmf-texlive/tex/latex/base/latin1.def
File: latin1.def 2006/05/05 v1.1b Input encoding file
))

LaTeX Warning: You have requested, on input line 89, version
               `2008/03/30' of package inputenc,
               but only version
               `2006/05/05 v1.1b Input encoding file'
               is available.

(/usr/share/texmf-texlive/tex/latex/base/ifthen.sty
Package: ifthen 2001/05/26 v1.1c Standard LaTeX ifthen package (DPC)
) (/usr/share/texmf-texlive/tex/latex/amsmath/amsmath.sty
Package: amsmath 2000/07/18 v2.13 AMS math features
\@mathmargin=\skip43
For additional information on amsmath, use the `?' option.
(/usr/share/texmf-texlive/tex/latex/amsmath/amstext.sty
Package: amstext 2000/06/29 v2.01
(/usr/share/texmf-texlive/tex/latex/amsmath/amsgen.sty
File: amsgen.sty 1999/11/30 v2.0
\@emptytoks=\toks16
\ex@=\dimen103
)) (/usr/share/texmf-texlive/tex/latex/amsmath/amsbsy.sty
Package: amsbsy 1999/11/29 v1.2d
\pmbraise@=\dimen104
) (/usr/share/texmf-texlive/tex/latex/amsmath/amsopn.sty
Package: amsopn 1999/12/14 v2.01 operator names
)
\inf@bad=\count88
LaTeX Info: Redefining \frac on input line 211.
\uproot@=\count89
\leftroot@=\count90
LaTeX Info: Redefining \overline on input line 307.
\classnum@=\count91
\DOTSCASE@=\count92
LaTeX Info: Redefining \ldots on input line 379.
LaTeX Info: Redefining \dots on input line 382.
LaTeX Info: Redefining \cdots on input line 467.
\Mathstrutbox@=\box26
\strutbox@=\box27
\big@size=\dimen105
LaTeX Font Info:    Redeclaring font encoding OML on input line 567.
LaTeX Font Info:    Redeclaring font encoding OMS on input line 568.
\macc@depth=\count93
\c@MaxMatrixCols=\count94
\dotsspace@=\muskip10
\c@parentequation=\count95
\dspbrk@lvl=\count96
\tag@help=\toks17
\row@=\count97
\column@=\count98
\maxfields@=\count99
\andhelp@=\toks18
\eqnshift@=\dimen106
\alignsep@=\dimen107
\tagshift@=\dimen108
\tagwidth@=\dimen109
\totwidth@=\dimen110
\lineht@=\dimen111
\@envbody=\toks19
\multlinegap=\skip44
\multlinetaggap=\skip45
\mathdisplay@stack=\toks20
LaTeX Info: Redefining \[ on input line 2666.
LaTeX Info: Redefining \] on input line 2667.
) (/usr/share/texmf-texlive/tex/latex/amsfonts/amssymb.sty
Package: amssymb 2002/01/22 v2.2d
(/usr/share/texmf-texlive/tex/latex/amsfonts/amsfonts.sty
Package: amsfonts 2001/10/25 v2.2f
\symAMSa=\mathgroup4
\symAMSb=\mathgroup5
LaTeX Font Info:    Overwriting math alphabet `\mathfrak' in version `bold'
(Font)                  U/euf/m/n --> U/euf/b/n on input line 132.
))

LaTeX Warning: You have requested, on input line 94, version
               `2009/06/22' of package amssymb,
               but only version
               `2002/01/22 v2.2d'
               is available.

(/usr/share/texmf-texlive/tex/latex/base/alltt.sty
Package: alltt 1997/06/16 v2.0g defines alltt environment
) (/usr/share/texmf-texlive/tex/latex/caption/caption.sty
Package: caption 2007/01/07 v3.0k Customising captions (AR)
(/usr/share/texmf-texlive/tex/latex/caption/caption3.sty
Package: caption3 2007/01/07 v3.0k caption3 kernel (AR)
(/usr/share/texmf-texlive/tex/latex/graphics/keyval.sty
Package: keyval 1999/03/16 v1.13 key=value parser (DPC)
\KV@toks@=\toks21
)
\captionmargin=\dimen112
\captionmarginx=\dimen113
\captionwidth=\dimen114
\captionindent=\dimen115
\captionparindent=\dimen116
\captionhangindent=\dimen117
)) (/usr/share/texmf-texlive/tex/latex/tools/array.sty
Package: array 2005/08/23 v2.4b Tabular extension package (FMi)
\col@sep=\dimen118
\extrarowheight=\dimen119
\NC@list=\toks22
\extratabsurround=\skip46
\backup@length=\skip47
)

LaTeX Warning: You have requested, on input line 99, version
               `2008/09/09' of package array,
               but only version
               `2005/08/23 v2.4b Tabular extension package (FMi)'
               is available.

(/usr/share/texmf-texlive/tex/latex/fancyhdr/fancyhdr.sty
\fancy@headwidth=\skip48
\f@ncyO@elh=\skip49
\f@ncyO@erh=\skip50
\f@ncyO@olh=\skip51
\f@ncyO@orh=\skip52
\f@ncyO@elf=\skip53
\f@ncyO@erf=\skip54
\f@ncyO@olf=\skip55
\f@ncyO@orf=\skip56
) (/usr/share/texmf-texlive/tex/latex/geometry/geometry.sty
Package: geometry 2002/07/08 v3.2 Page Geometry
\Gm@cnth=\count100
\Gm@cntv=\count101
\c@Gm@tempcnt=\count102
\Gm@bindingoffset=\dimen120
\Gm@wd@mp=\dimen121
\Gm@odd@mp=\dimen122
\Gm@even@mp=\dimen123
\Gm@dimlist=\toks23
(/usr/share/texmf-texlive/tex/xelatex/xetexconfig/geometry.cfg))

LaTeX Warning: You have requested, on input line 165, version
               `2010/09/12' of package geometry,
               but only version
               `2002/07/08 v3.2 Page Geometry'
               is available.

(/usr/share/texmf-texlive/tex/latex/hyperref/hyperref.sty
Package: hyperref 2007/02/07 v6.75r Hypertext links for LaTeX
\@linkdim=\dimen124
\Hy@linkcounter=\count103
\Hy@pagecounter=\count104
(/usr/share/texmf-texlive/tex/latex/hyperref/pd1enc.def
File: pd1enc.def 2007/02/07 v6.75r Hyperref: PDFDocEncoding definition (HO)
) (/etc/texmf/tex/latex/config/hyperref.cfg
File: hyperref.cfg 2002/06/06 v1.2 hyperref configuration of TeXLive
) (/usr/share/texmf-texlive/tex/latex/oberdiek/kvoptions.sty
Package: kvoptions 2006/08/22 v2.4 Connects package keyval with LaTeX options (
HO)
)
Package hyperref Info: Option `hyperfootnotes' set `false' on input line 2238.
Package hyperref Info: Option `bookmarks' set `true' on input line 2238.
Package hyperref Info: Option `linktocpage' set `false' on input line 2238.
Package hyperref Info: Option `pdfdisplaydoctitle' set `true' on input line 223
8.
Package hyperref Info: Option `pdfpagelabels' set `true' on input line 2238.
Package hyperref Info: Option `bookmarksopen' set `true' on input line 2238.
Package hyperref Info: Option `colorlinks' set `true' on input line 2238.
Package hyperref Info: Hyper figures OFF on input line 2288.
Package hyperref Info: Link nesting OFF on input line 2293.
Package hyperref Info: Hyper index ON on input line 2296.
Package hyperref Info: Plain pages OFF on input line 2303.
Package hyperref Info: Backreferencing OFF on input line 2308.
Implicit mode ON; LaTeX internals redefined
Package hyperref Info: Bookmarks ON on input line 2444.
(/usr/share/texmf-texlive/tex/latex/ltxmisc/url.sty
\Urlmuskip=\muskip11
Package: url 2005/06/27  ver 3.2  Verb mode for urls, etc.
)
LaTeX Info: Redefining \url on input line 2599.
\Fld@menulength=\count105
\Field@Width=\dimen125
\Fld@charsize=\dimen126
\Choice@toks=\toks24
\Field@toks=\toks25
Package hyperref Info: Hyper figures OFF on input line 3102.
Package hyperref Info: Link nesting OFF on input line 3107.
Package hyperref Info: Hyper index ON on input line 3110.
Package hyperref Info: backreferencing OFF on input line 3117.
Package hyperref Info: Link coloring ON on input line 3120.
\Hy@abspage=\count106
\c@Item=\count107
)
*hyperref using driver hpdftex*
(/usr/share/texmf-texlive/tex/latex/hyperref/hpdftex.def
File: hpdftex.def 2007/02/07 v6.75r Hyperref driver for pdfTeX
\Fld@listcount=\count108
)

LaTeX Warning: You have requested, on input line 185, version
               `2011/04/17' of package hyperref,
               but only version
               `2007/02/07 v6.75r Hypertext links for LaTeX'
               is available.

\TmpLen=\skip57
(./37030-t.aux)
\openout1 = `37030-t.aux'.

LaTeX Font Info:    Checking defaults for OML/cmm/m/it on input line 271.
LaTeX Font Info:    ... okay on input line 271.
LaTeX Font Info:    Checking defaults for T1/cmr/m/n on input line 271.
LaTeX Font Info:    ... okay on input line 271.
LaTeX Font Info:    Checking defaults for OT1/cmr/m/n on input line 271.
LaTeX Font Info:    ... okay on input line 271.
LaTeX Font Info:    Checking defaults for OMS/cmsy/m/n on input line 271.
LaTeX Font Info:    ... okay on input line 271.
LaTeX Font Info:    Checking defaults for OMX/cmex/m/n on input line 271.
LaTeX Font Info:    ... okay on input line 271.
LaTeX Font Info:    Checking defaults for U/cmr/m/n on input line 271.
LaTeX Font Info:    ... okay on input line 271.
LaTeX Font Info:    Checking defaults for PD1/pdf/m/n on input line 271.
LaTeX Font Info:    ... okay on input line 271.
(/usr/share/texmf-texlive/tex/latex/ragged2e/ragged2e.sty
Package: ragged2e 2003/03/25 v2.04 ragged2e Package (MS)
(/usr/share/texmf-texlive/tex/latex/everysel/everysel.sty
Package: everysel 1999/06/08 v1.03 EverySelectfont Package (MS)
LaTeX Info: Redefining \selectfont on input line 125.
)
\CenteringLeftskip=\skip58
\RaggedLeftLeftskip=\skip59
\RaggedRightLeftskip=\skip60
\CenteringRightskip=\skip61
\RaggedLeftRightskip=\skip62
\RaggedRightRightskip=\skip63
\CenteringParfillskip=\skip64
\RaggedLeftParfillskip=\skip65
\RaggedRightParfillskip=\skip66
\JustifyingParfillskip=\skip67
\CenteringParindent=\skip68
\RaggedLeftParindent=\skip69
\RaggedRightParindent=\skip70
\JustifyingParindent=\skip71
)
Package caption Info: hyperref package v6.74m (or newer) detected on input line
 271.
-------------------- Geometry parameters
paper: class default
landscape: --
twocolumn: --
twoside: true
asymmetric: --
h-parts: 9.03374pt, 325.215pt, 9.03375pt
v-parts: 4.15848pt, 495.49379pt, 6.23773pt
hmarginratio: 1:1
vmarginratio: 2:3
lines: --
heightrounded: --
bindingoffset: 0.0pt
truedimen: --
includehead: true
includefoot: true
includemp: --
driver: pdftex
-------------------- Page layout dimensions and switches
\paperwidth  343.28249pt
\paperheight 505.89pt
\textwidth  325.215pt
\textheight 433.62pt
\oddsidemargin  -63.23625pt
\evensidemargin -63.23624pt
\topmargin  -68.11151pt
\headheight 12.0pt
\headsep    19.8738pt
\footskip   30.0pt
\marginparwidth 98.0pt
\marginparsep   7.0pt
\columnsep  10.0pt
\skip\footins  10.8pt plus 4.0pt minus 2.0pt
\hoffset 0.0pt
\voffset 0.0pt
\mag 1000
\@twosidetrue \@mparswitchtrue 
(1in=72.27pt, 1cm=28.45pt)
-----------------------
(/usr/share/texmf-texlive/tex/latex/graphics/color.sty
Package: color 2005/11/14 v1.0j Standard LaTeX Color (DPC)
(/etc/texmf/tex/latex/config/color.cfg
File: color.cfg 2007/01/18 v1.5 color configuration of teTeX/TeXLive
)
Package color Info: Driver file: pdftex.def on input line 130.
(/usr/share/texmf-texlive/tex/latex/pdftex-def/pdftex.def
File: pdftex.def 2007/01/08 v0.04d Graphics/color for pdfTeX
\Gread@gobject=\count109
(/usr/share/texmf/tex/context/base/supp-pdf.tex
[Loading MPS to PDF converter (version 2006.09.02).]
\scratchcounter=\count110
\scratchdimen=\dimen127
\scratchbox=\box28
\nofMPsegments=\count111
\nofMParguments=\count112
\everyMPshowfont=\toks26
\MPscratchCnt=\count113
\MPscratchDim=\dimen128
\MPnumerator=\count114
\everyMPtoPDFconversion=\toks27
)))
Package hyperref Info: Link coloring ON on input line 271.
(/usr/share/texmf-texlive/tex/latex/hyperref/nameref.sty
Package: nameref 2006/12/27 v2.28 Cross-referencing by name of section
(/usr/share/texmf-texlive/tex/latex/oberdiek/refcount.sty
Package: refcount 2006/02/20 v3.0 Data extraction from references (HO)
)
\c@section@level=\count115
)
LaTeX Info: Redefining \ref on input line 271.
LaTeX Info: Redefining \pageref on input line 271.
(./37030-t.out) (./37030-t.out)
\@outlinefile=\write3
\openout3 = `37030-t.out'.


Overfull \hbox (44.54031pt too wide) in paragraph at lines 284--284
[]\OT1/cmtt/m/n/8 Title: Some Famous Problems of the Theory of Numbers and in p
articular Waring's Problem[] 
 []


Overfull \hbox (23.29001pt too wide) in paragraph at lines 295--295
[]\OT1/cmtt/m/n/8 *** START OF THIS PROJECT GUTENBERG EBOOK FAMOUS PROBLEMS OF 
THEORY OF NUMBERS ***[] 
 []

[1

{/var/lib/texmf/fonts/map/pdftex/updmap/pdftex.map}]
LaTeX Font Info:    Try loading font information for U+msa on input line 310.
(/usr/share/texmf-texlive/tex/latex/amsfonts/umsa.fd
File: umsa.fd 2002/01/19 v2.2g AMS font definitions
)
LaTeX Font Info:    Try loading font information for U+msb on input line 310.
(/usr/share/texmf-texlive/tex/latex/amsfonts/umsb.fd
File: umsb.fd 2002/01/19 v2.2g AMS font definitions
) [2

] [1



] [2] [1



] [2] [3] [4] [5] [6] [7] [8] [9] [10] [11] [12] [13] [14] [15] [16] [17] [18] 
[19] [20]
Overfull \hbox (5.66283pt too wide) detected at line 1250
[]
 []

[21] [22] [23] [24] [25] [26] [27] [28]
Overfull \hbox (6.59698pt too wide) detected at line 1575
[]
 []

[29] [30] [31] [32] [33



] [34]
Overfull \hbox (14.78989pt too wide) in paragraph at lines 1690--1690
[]\OT1/cmtt/m/n/8 *** END OF THIS PROJECT GUTENBERG EBOOK FAMOUS PROBLEMS OF TH
EORY OF NUMBERS ***[] 
 []

[35] [36] [37] [38] [39] [40] [41] [42] (./37030-t.aux)

 *File List*
    book.cls    2005/09/16 v1.4f Standard LaTeX document class
    bk12.clo    2005/09/16 v1.4f Standard LaTeX file (size option)
inputenc.sty    2006/05/05 v1.1b Input encoding file
  latin1.def    2006/05/05 v1.1b Input encoding file
  ifthen.sty    2001/05/26 v1.1c Standard LaTeX ifthen package (DPC)
 amsmath.sty    2000/07/18 v2.13 AMS math features
 amstext.sty    2000/06/29 v2.01
  amsgen.sty    1999/11/30 v2.0
  amsbsy.sty    1999/11/29 v1.2d
  amsopn.sty    1999/12/14 v2.01 operator names
 amssymb.sty    2002/01/22 v2.2d
amsfonts.sty    2001/10/25 v2.2f
   alltt.sty    1997/06/16 v2.0g defines alltt environment
 caption.sty    2007/01/07 v3.0k Customising captions (AR)
caption3.sty    2007/01/07 v3.0k caption3 kernel (AR)
  keyval.sty    1999/03/16 v1.13 key=value parser (DPC)
   array.sty    2005/08/23 v2.4b Tabular extension package (FMi)
fancyhdr.sty    
geometry.sty    2002/07/08 v3.2 Page Geometry
geometry.cfg
hyperref.sty    2007/02/07 v6.75r Hypertext links for LaTeX
  pd1enc.def    2007/02/07 v6.75r Hyperref: PDFDocEncoding definition (HO)
hyperref.cfg    2002/06/06 v1.2 hyperref configuration of TeXLive
kvoptions.sty    2006/08/22 v2.4 Connects package keyval with LaTeX options (HO
)
     url.sty    2005/06/27  ver 3.2  Verb mode for urls, etc.
 hpdftex.def    2007/02/07 v6.75r Hyperref driver for pdfTeX
ragged2e.sty    2003/03/25 v2.04 ragged2e Package (MS)
everysel.sty    1999/06/08 v1.03 EverySelectfont Package (MS)
   color.sty    2005/11/14 v1.0j Standard LaTeX Color (DPC)
   color.cfg    2007/01/18 v1.5 color configuration of teTeX/TeXLive
  pdftex.def    2007/01/08 v0.04d Graphics/color for pdfTeX
supp-pdf.tex
 nameref.sty    2006/12/27 v2.28 Cross-referencing by name of section
refcount.sty    2006/02/20 v3.0 Data extraction from references (HO)
 37030-t.out
 37030-t.out
    umsa.fd    2002/01/19 v2.2g AMS font definitions
    umsb.fd    2002/01/19 v2.2g AMS font definitions
 ***********

 ) 
Here is how much of TeX's memory you used:
 4706 strings out of 94074
 64607 string characters out of 1165154
 126581 words of memory out of 1500000
 7862 multiletter control sequences out of 10000+50000
 17458 words of font info for 66 fonts, out of 1200000 for 2000
 645 hyphenation exceptions out of 8191
 35i,13n,43p,258b,484s stack positions out of 5000i,500n,6000p,200000b,5000s
</usr/share/texmf-texlive/fonts/type1/bluesky/cm/cmbx10.pfb></usr/share/texmf
-texlive/fonts/type1/bluesky/cm/cmbx12.pfb></usr/share/texmf-texlive/fonts/type
1/bluesky/cm/cmcsc10.pfb></usr/share/texmf-texlive/fonts/type1/bluesky/cm/cmex1
0.pfb></usr/share/texmf-texlive/fonts/type1/bluesky/cm/cmmi10.pfb></usr/share/t
exmf-texlive/fonts/type1/bluesky/cm/cmmi12.pfb></usr/share/texmf-texlive/fonts/
type1/bluesky/cm/cmmi6.pfb></usr/share/texmf-texlive/fonts/type1/bluesky/cm/cmm
i8.pfb></usr/share/texmf-texlive/fonts/type1/bluesky/cm/cmr10.pfb></usr/share/t
exmf-texlive/fonts/type1/bluesky/cm/cmr12.pfb></usr/share/texmf-texlive/fonts/t
ype1/bluesky/cm/cmr17.pfb></usr/share/texmf-texlive/fonts/type1/bluesky/cm/cmr6
.pfb></usr/share/texmf-texlive/fonts/type1/bluesky/cm/cmr7.pfb></usr/share/texm
f-texlive/fonts/type1/bluesky/cm/cmr8.pfb></usr/share/texmf-texlive/fonts/type1
/bluesky/cm/cmsy10.pfb></usr/share/texmf-texlive/fonts/type1/bluesky/cm/cmsy6.p
fb></usr/share/texmf-texlive/fonts/type1/bluesky/cm/cmsy8.pfb></usr/share/texmf
-texlive/fonts/type1/bluesky/cm/cmti10.pfb></usr/share/texmf-texlive/fonts/type
1/bluesky/cm/cmti8.pfb></usr/share/texmf-texlive/fonts/type1/bluesky/cm/cmtt8.p
fb></usr/share/texmf-texlive/fonts/type1/bluesky/ams/msam10.pfb>
Output written on 37030-t.pdf (46 pages, 247814 bytes).
PDF statistics:
 339 PDF objects out of 1000 (max. 8388607)
 66 named destinations out of 1000 (max. 131072)
 49 words of extra memory for PDF output out of 10000 (max. 10000000)

